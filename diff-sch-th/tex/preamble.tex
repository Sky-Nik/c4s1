\usepackage[utf8]{inputenc}
\usepackage[T1,T2A]{fontenc}
\usepackage[english,russian,ukrainian]{babel}
\usepackage{amsfonts}
\usepackage{amsmath}
\usepackage{amssymb}
\usepackage[usenames,svgnames,dvipsnames]{xcolor}
\usepackage[unicode=true]{hyperref}
\hypersetup{
    colorlinks,
    linkcolor={red!50!black},
    citecolor={green!50!black},
    urlcolor={blue!80!black}
}
\usepackage[nameinlink,ukrainian]{cleveref}
\usepackage{graphicx}
\usepackage{enumitem}
\usepackage{float}


\newcommand{\NN}{\mathbb{N}}
\newcommand{\ZZ}{\mathbb{Z}}
\newcommand{\QQ}{\mathbb{Q}}
\newcommand{\RR}{\mathbb{R}}
\newcommand{\CC}{\mathbb{C}}
\newcommand{\LL}{\mathcal{L}}
\newcommand{\inv}{^{-1}}
\newcommand{\conjugate}[1]{#1^\star}
\newcommand{\dconjugate}[1]{#1^{\star\star}}
\newcommand{\transpose}[1]{#1^\intercal}
\newcommand{\weakto}{\rightharpoonup}
\newcommand{\closure}[1]{\overline{#1}}
\newcommand{\half}{\frac{1}{2}}
\newcommand{\cbrt}[1]{\sqrt[3]{#1}}

\renewcommand{\emptyset}{\varnothing}
\renewcommand{\epsilon}{\varepsilon}
\renewcommand{\phi}{\varphi}

\DeclareMathOperator{\sign}{sgn}
\DeclareMathOperator{\spanning}{span}
\DeclareMathOperator{\kernel}{ker}
\DeclareMathOperator{\Closure}{cl}
\DeclareMathOperator{\Interior}{int}
\DeclareMathOperator{\Imag}{Im}
\DeclareMathOperator{\Real}{Re}
\DeclareMathOperator*{\Bigcup}{\bigcup}
\DeclareMathOperator*{\Bigcap}{\bigcap}
\DeclareMathOperator{\const}{const}

\newcommand*\diff{\mathop{}\!\mathrm{d}}
\newcommand{\shortLectureDescription}[1]{\small{\textit{#1}}\normalsize\medskip}
\newcommand{\shortHomeworkDescription}[1]{\small{\textit{#1}}\normalsize\medskip}


\makeatletter
\let\oldvec\vec
\renewcommand{\vec}[1]{\oldvec{#1}\@ifnextchar{^}{\,}{}}
\makeatother


\usepackage{amsthm}
\usepackage[dvipsnames]{xcolor}
\usepackage{thmtools}
\usepackage[framemethod=TikZ]{mdframed}

\theoremstyle{definition}

\mdfdefinestyle{mdbluebox}{%
	roundcorner = 10pt,
	linewidth=1pt,
	skipabove=12pt,
	innerbottommargin=9pt,
	skipbelow=2pt,
	nobreak=true,
	linecolor=blue,
	backgroundcolor=TealBlue!5,
}

\declaretheoremstyle[
	headfont=\sffamily\bfseries\color{MidnightBlue},
	mdframed={style=mdbluebox},
	headpunct={\\[3pt]},
	postheadspace={0pt}
]{thmbluebox}

\declaretheorem[style=thmbluebox,numberwithin=chapter,name=Теорема]{theorem}
\declaretheorem[style=thmbluebox,sibling=theorem,name=Лема]{lemma}
\declaretheorem[style=thmbluebox,sibling=theorem,name=Твердження]{proposition}
\declaretheorem[style=thmbluebox,sibling=theorem,name=Формула]{th_formula}
\declaretheorem[style=thmbluebox,sibling=theorem,name=Закон]{th_law}
\declaretheorem[style=thmbluebox,sibling=theorem,name=Факт]{fact}
\declaretheorem[style=thmbluebox,sibling=theorem,name=Наслідок]{corollary}

\mdfdefinestyle{mdredbox}{%
	linewidth=0.5pt,
	skipabove=12pt,
	frametitleaboveskip=5pt,
	frametitlebelowskip=0pt,
	skipbelow=2pt,
	frametitlefont=\bfseries,
	innertopmargin=4pt,
	innerbottommargin=8pt,
	nobreak=true,
	linecolor=RawSienna,
	backgroundcolor=Salmon!5,
}

\declaretheoremstyle[
	headfont=\bfseries\color{RawSienna},
	mdframed={style=mdredbox},
	headpunct={\\[3pt]},
	postheadspace={0pt},
]{thmredbox}

\declaretheorem[style=thmredbox,name=Приклад,sibling=theorem]{example}

\mdfdefinestyle{mdgreenbox}{%
	skipabove=8pt,
	linewidth=2pt,
	rightline=false,
	leftline=true,
	topline=false,
	bottomline=false,
	linecolor=ForestGreen,
	backgroundcolor=ForestGreen!5,
}

\declaretheoremstyle[
	headfont=\bfseries\sffamily\color{ForestGreen!70!black},
	bodyfont=\normalfont,
	spaceabove=2pt,
	spacebelow=1pt,
	mdframed={style=mdgreenbox},
	headpunct={ --- },
]{thmgreenbox}

\declaretheorem[name=Зауваження,sibling=theorem,style=thmgreenbox]{remark}

\mdfdefinestyle{mdblackbox}{%
	skipabove=8pt,
	linewidth=3pt,
	rightline=false,
	leftline=true,
	topline=false,
	bottomline=false,
	linecolor=black,
	backgroundcolor=RedViolet!5!gray!5,
}

\declaretheoremstyle[
	headfont=\bfseries,
	bodyfont=\normalfont\small,
	spaceabove=0pt,
	spacebelow=0pt,
	mdframed={style=mdblackbox}
]{thmblackbox}

\declaretheorem[style=thmblackbox,numberwithin=chapter,name=Запитання]{ques}
\declaretheorem[style=thmblackbox,numberwithin=chapter,name=Вправа]{exercise}
\declaretheorem[style=thmblackbox,numberwithin=chapter,name=Означення]{definition}

\usepackage[skip=\smallskipamount,indent=\parindent]{parskip}
\usepackage[headsepline]{scrlayer-scrpage}
\renewcommand{\headfont}{}
\addtolength{\textheight}{3.14cm}
\setlength{\footskip}{0.5in}
\setlength{\headsep}{10pt}
\automark[chapter]{chapter}
\rohead{\footnotesize\thepage}
\makeatletter
\rehead{\footnotesize\textbf{\sffamily \@title}, \emph{\@author}}
\makeatother
\lehead{\footnotesize\thepage}
\lohead{\footnotesize\leftmark}
\chead{}
\rofoot{}
\refoot{}
\lefoot{}
\lofoot{}

\makeatletter
\usepackage{etoolbox}
\pretocmd{\tableofcontents}{%
  \if@openright\cleardoublepage\else\clearpage\fi
  \pdfbookmark[0]{\contentsname}{toc}%
}{}{}%
\makeatother
\setcounter{tocdepth}{1}
\usepackage[tocindentauto]{tocstyle}
\usetocstyle{KOMAlike}

\usepackage[tight]{minitoc}
\mtcsetfont{parttoc}{chapter}{\sffamily\bfseries}
\mtcsetfont{parttoc}{section}{\footnotesize\rmfamily\upshape\mdseries}
\mtcsetfont{parttoc}{subsection}{\footnotesize\rmfamily\upshape\mdseries}
\setcounter{parttocdepth}{1}
\renewcommand*{\partheadstartvskip}{\vspace*{20em}}
\renewcommand*{\partheadendvskip}{}
\renewcommand\beforeparttoc{\noindent{\bfseries \Large Частина \thepart: Зміст}}
\doparttoc[n]

\renewcommand*{\sectionformat}{\color{purple}\S\thesection\autodot\enskip}
\renewcommand*{\subsectionformat}{\color{purple}\S\thesubsection\autodot\enskip}
\renewcommand{\thesubsection}{\thesection.\roman{subsection}}

\addtokomafont{chapterprefix}{\raggedleft}
\RedeclareSectionCommand[beforeskip=0.5em]{chapter}
\renewcommand*{\chapterformat}{%
\mbox{\scalebox{1.5}{\chapappifchapterprefix{\nobreakspace}}%
\scalebox{2.718}{\color{purple}\thechapter\autodot}\enskip}}

\addtokomafont{partprefix}{\rmfamily}
\renewcommand*{\partformat}{\color{purple}\scalebox{2.5}{\thepart}}

\newcommand{\listhack}{$\empty$}
\newcommand{\nothing}{$\left.\right.$}

\section{Контрольні запитання}
\begin{enumerate}
    \item Мета, задачі і проблеми чисельного моделювання. Чисельне моделювання процесів гідродинаміки. Основні закони та рівняння. Дивергентна і недивергентна форми запису системи рівнянь Нав'є---Стокса. Системи обезрозмірювання рівнянь. 

    \item Методи апроксимації диференціальних операторів (загальна характеристика). Методи засновані на використанні інтерполяційних многочленів, ітегро-інтерпо\-ляційний метод та метод контрольного об'єму. Основні властивості різницевих схем: консервативність, транспортивність, дисипативність.

    \item Поняття точного та узагальненого розв'язків диференціальних та різницевих рівнянь в банахових просторах. Розв'язуючі оператори. Коректність постановок задач. Збіжність та стійкість Терема Лакса. 

    \item Поняття стійкості та збіжності різницевих схем. 

    \item Поняття багатошарових схем та багатокрокових алгоритмів. Методи їх дослідження. Особливості подання початкових умов. Зведення їх до одношарових.

    \item Метод дискретних збурень.

    \item Метод фон Неймана. Коефіцієнт та матриця переходу. Явні та неявні різницеві схеми: алгоритм дослідження та теореми про збіжність ряду Фур'є та стійкість явної та неявної різницевих схем.

    \item Стійкість різницевих схем, які апроксимують рівняння дифузійного та конвективного переносу, дослідженння консервативності та транспортивності.

    \item Дослідження стійкості чисельних алгоритмів із змінними операторними коефіцієнтами Енергетичний метод.

    \item Принцип заморожених коефіцієнтів та ознака Бабенка---Гельфонда. 

    \item Метод першого диференціального наближення. Приклади застосування методу диференціального наближення. Гіперболічна та параболічна форми запису першого диференціального наближення. Схемна в'язкість.
\end{enumerate}

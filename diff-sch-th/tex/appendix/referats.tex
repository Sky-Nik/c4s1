\section{Теми для рефератів}
\begin{enumerate}
    \item Апроксимація оператора конвективного переносу. [1, 5, 8, 9] \medskip 
    
    \textit{Властивості операторів ковективного переносу. Апроксимація оператора конвективного переносу. Схеми з несиметричними різницями за просторовими координатами. Транспортивність чисельних моделей. Зв'язок між консервативністю та транспортивністю.}
    
    \item Схеми з різницями проти потоку. Схеми з донорними комірками. Консервативність та транспортивність схем з різницями проти потоку та схем з донорними комірками. [1]
    
    \item Фазова похибка різницевих схем.  [1, 2] \medskip 
    
    \textit{Схема Лейта. Фазова похибка. Її виникнення, розповсюдження і методи зменшення.}
    
    \item Аналіз виникнення фазової похибки при апроксимації часових та просторових диференціальниї операторві. [1, 2]
    
    \item Тришарові різницеві схеми. [2, 3] \medskip 
    
    \textit{Дослідження стійкості та збіжності. Схема Дюфорта---Франклена.}
    
    \item Апроксимація рівняння конвективного переносу у двовимірному просторі при виконанні умови нерозривності. [8, 9]
    
    \item Схеми методу змінних напрямків.  \medskip 
    
    \textit{Явні та неявні схеми методу змінних напрямків. \\ Схеми Пісмена---Решфорда, Яненка, Брили, Саульєва.}

    \item Схеми методу знінних напрямків Адамса---Бешфорта та Крокко. Схема Лейта. [1]

    \item Загальні схеми методу розщеплення. [8] \medskip 
    
    \textit{Метод дробових кроків М.М. Яненка, метод розщеплення Г.І. Марчука. \\ Їх схожість і відмінності.}

    \item Практична побудова багатокрокових явних схем. \medskip 
    
    \textit{Багатокрокові явні схеми Лакса, Лакса---Вендроффа, Мак---Кормака та Браіловської.}

    \item Дослідження умови стійкості явних схем Лакса та Лакса---Вендроффа. [2, 10]

    \item Загальна ідея багатокрокових алгоритмів. Двокрокові алгоритми. ДС-алгоритми. [1, 4, 5] \medskip 
    
    \textit{Уточнюючі багатокрокові алгоритми. Схеми предиктор-коректор. Ідея двокрокового симетризованого методу, дослідження стійкості та реалізація.}

    \item Тришарові різницеві схеми другого та третього порядків. [4]

    \item Економічні ітераційні багато шарові алгоритми. \medskip 
    
    \textit{Методи прискорення збіжності ітераційних алгоритмів. Дослідження кількості операцій в одному циклі ітераційного алгоритму. Асимптотична та практична збіжність.}

    \item Ітераційні алгоритми для знаходження розв'язку різницевої задачі Діріхле та першої крайової задачі рівняння повного переносу. [5, 6]
    
    \item Різницеві схеми для системи телеграфних рівняння. Дослідження схемної в'яз\-кос\-ті.
\end{enumerate}

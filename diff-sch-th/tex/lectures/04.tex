\chapter{Багатошарові та багатокрокові чисельні моделі}

\shortLectureDescription{Багатошарові та багатокрокові чисельні моделі. Методи дослідження. Особливості подання початкових умов. Зведення їх до одношарових.}

\section{Багатошарові чисельні моделі}

Нехай крайова задача
\begin{equation}
    \label{eq:l4.1}
    \frac{\partial \vec u(t)}{\partial t} = A \vec u, \quad t > 0
\end{equation}
апроксимується за допомогою $(q+1)$-шарової різницевої схеми
\begin{equation}
    \label{eq:l4.2}
    B_q \vec u^{n + q} + B_{q - 1} \vec u^{n + q - 1} + \ldots + B_0 \vec u^n = 0,
\end{equation}
де $B_q, \ldots, B_0$ --- скінченно---різницеві оператори, а $\vec u^n$ --- деяка апроксимація $\vec u(n \Delta t)$. Припустимо, що рівняння \eqref{eq:l4.2} разом з відповідними граничними умовами має єдиний розв'язок відносно $\vec u^{n + q}$, залежний від $\vec u^{n + q - 1}, \ldots, \vec u^n$. Запишемо еквівалентне до \eqref{eq:l4.2} рівняння 
\begin{equation}
    \label{eq:l4.3}
    \vec u^{n + q} = C_{q - 1} \vec u^{n + q - 1} + \ldots + C_0 \vec u^n
\end{equation}
де $C_i = C_i(\Delta t) = B_q^{-1} B_i$ --- лінійні обмежені оператори, а $\Delta x_j$ є заданими функціями від $\Delta t$. \medskip

Уведемо нові залежні змінні: $\vec v^n = \vec u^{n - 1}$, $\vec w^n = \vec v^{n - 1}$ і так далі, тоді рівняння \eqref{eq:l4.3} замінено системою 
\begin{equation}
    \label{eq:l4.4}
    \vec u^{n + 1} = C_{q - 1} \vec u^n + C_{q - 2} \vec v^n + C_{q - 3} \vec w^n + \ldots, \quad \vec v^{n + 1} = \vec u^n, \vec w^{n + 1} = \vec v^n, \ldots
\end{equation}

Якщо $\tilde B$ це простір, елементами якого є $q$-вимірні вектор-стовпчики
\begin{equation*}
    \vec{\tilde{\phi}} = \begin{pmatrix} \vec u^{n + q - 1} & \cdots & \vec u^n \end{pmatrix}^\intercal
\end{equation*}
з нормою введеною, наприклад, так:
\begin{equation*}
    \left\| \begin{pmatrix} \vec u & \vec v & \cdots & \end{pmatrix}^\intercal \right\| = \sqrt{\|\vec u\|^2 + \|\vec v\|^2 + \ldots}
\end{equation*}
і якщо покласти
\begin{equation*}
    \tilde C = \begin{pmatrix}
        C_{q - 1} & C_{q - 2} & \cdots & \cdots & C_1 & C_0 \\
        E & 0 & \cdots & \cdots & \cdots & 0 \\
        0 & \ddots & \ddots & & & \vdots \\
        \vdots & \ddots & \ddots & \ddots & & \vdots \\
        \vdots & & \ddots & \ddots & \ddots & \vdots \\
        0 & \cdots & \cdots & 0 & E & 0
    \end{pmatrix},
\end{equation*}
де $E$ --- тотожний оператор, то рівняння \eqref{eq:l4.4} можна записати так:
\begin{equation}
    \label{eq:l4.5}
    \vec{\tilde{\phi}}^{n + 1} = \tilde C(\Delta t) \vec{\tilde{\phi}}^n,
\end{equation}
де 
\begin{equation}
    \vec{\tilde{\phi}}^n = \begin{pmatrix} \vec u^{n + q} & \cdots & \vec u^n \end{pmatrix}^\intercal.
\end{equation}

Тут $\tilde C$ --- матриця, яка відповідає системі \eqref{eq:l4.4}. \medskip

Тобто $(q + 1)$-шарова різницева задача \eqref{eq:l4.3} зводиться до системи двошарових схем.

\section{Проблема постановки початкових умов для багатошарових різницевих схем}

Початкові дані для диференціальної задачі задаються в момент часу $t_0$, а дані для різницевої задачі в точках сітки повинні бути визначені через початкові дані за допомогою апроксимації, яка дозволяє починати обчислення розв'язку за рівнянням \eqref{eq:l4.3}. Припустимо, що значення $\vec{\tilde{\phi}}$ обчислені точно, тобто 
\begin{equation}
    \vec{\tilde{\phi}} = \begin{pmatrix} E((q - 1)\Delta t) \vec u^0 & E((q - 2)\Delta t) \vec u^0 & \cdots & \vec u^0 \end{pmatrix}^\intercal,
\end{equation}
де $\vec u^0 \in B$, а $E(t)$ --- розв'язуючий оператор для \eqref{eq:l4.1}. Тоді 
\begin{equation}
    \tilde{\vec{u}}^0 = \begin{pmatrix} \vec u^0 & \cdots & \vec u^0 \end{pmatrix}^\intercal,
\end{equation}
а
\begin{equation}
    \tilde S = \begin{pmatrix}
        E((q - 1) \Delta t) & 0 & \cdots & \cdots & 0 \\
        0 & E((q - 2) \Delta t) & \ddots & & \vdots \\
        \vdots & \ddots & \ddots & \ddots & \vdots \\
        \vdots & & \ddots & \ddots & 0 \\
        0 & \cdots & \cdots & 0 & E
    \end{pmatrix}
\end{equation}
---	діагональна матриця. \medskip

Отже, наближений розв'язок крайової задачі \eqref{eq:l4.1} визначається формулою 
\begin{equation}
    \vec{\tilde{\phi}} = \tilde C(\Delta t) \tilde S \tilde{\vec{u}}^0,
\end{equation}
яка є апроксимацією виразу
\begin{equation}
    \begin{pmatrix} \vec u ((n + q - 1) \Delta t) & \vec u ((n + q - 2) \Delta t) & \cdots & \vec u (n \Delta t) \end{pmatrix}^\intercal = E (n \Delta t) \tilde S \tilde{\vec{u}}^0.
\end{equation}

Порівняємо $\tilde C(\Delta t)^n \tilde S \tilde{\vec{u}}^0$ з $E (n \Delta t) \tilde S \tilde{\vec{u}}^0$. При $\Delta t \to 0$ приходимо до підпростору $\tilde B^0$ простору $\tilde B$, елементами якого є вектори, всі $q$ компонент яких рівні між собою. Тоді
\begin{equation}
    \begin{pmatrix} \vec u(t) & \cdots & \vec u(t) \end{pmatrix}^\intercal = E(t) \begin{pmatrix} \vec u^0 & \cdots & \vec u^0 \end{pmatrix}^\intercal
\end{equation}

До цієї ж границі прямує і $E (n \Delta t) \tilde S \tilde{\vec{u}}^0$ при $\Delta t \to 0$ ($\tilde S \to E$). Елементи $\vec u(t)$ і $\vec u^0$ належать простору $B^0$, але їх апроксимації --- ні, вони лежать поза цим простором. При практичних обчисленнях, початкові значення отримуємо не з допомогою обчислення $\vec{\tilde{\phi}}^0 = \tilde S \tilde{\vec{u}}^0$, а з допомогою апроксимації $\vec{\tilde{\phi}}^0 = \vec \psi$. Для того, щоб наближене значення розв'язку $\tilde{\vec{u}}^n$ збігалось до точного розв'язку, необхідно, щоб $\vec u^n \to \vec u(t)$, а $\vec{\tilde{\phi}}^0 = \vec \psi \to \vec u(0)$ при $\Delta t \to 0$ незалежно одне від одного. 

\section{Властивості багатошарових різницевих схем}

\begin{definition}
    Багатошарову різницеву схему \eqref{eq:l4.5} називають \textit{стійкою}, якщо для деяко-го додатного $\tau$ при $0 < \Delta t < \tau$ і $0 \le n \Delta t \le T$ сім'я операторів $\tilde C^n(\Delta t)$ з \eqref{eq:l4.5} рівномірно обмежена $\forall n$.
\end{definition}

\begin{definition}
    Важливу роль відіграють \textit{умови узгодженості}, які полягають в тому, що існує щільна в просторі $B$ множина $U^0$ можливих початкових елементів для точних розв'язків така, що якщо $\vec u^0 \in U^0$ та існує $\epsilon > 0$, таке що, якщо покласти 
    \begin{equation}
        \tilde{\vec{u}}^0 = \begin{pmatrix} E(t) \vec u^0 & \cdots E(t) \vec u^0 \end{pmatrix}
    \end{equation}
    то 
    \begin{equation}
        \label{eq:l4.6}
        \| (\tilde C(\Delta t) - E(\Delta t) \tilde S) \vec{\tilde{u}}(t) \| < \epsilon \Delta t
    \end{equation}
    для достатньо малих $\Delta t$ і $0 \le t \le T$.
\end{definition}

\begin{definition}
    Апроксимація є \textit{збіжною}, якщо для довільних послідовностей $\Delta_j t$ і $n_j$, таких що $\Delta_j t > 0$ і $n_j \Delta_j t \to t$ при $j \to \infty$, ($0 \le t \le T$) та $\forall \tilde{\vec{u}} \in \tilde B^0$ буде
    \begin{equation}
        \lim_{\substack{j \to \infty \\ \vec \psi \to \vec u^0}} \left\| \tilde C(\Delta_j t)^{n_j} \vec \psi - E(t) \vec{\tilde{u}}(t) \right\| = 0.
    \end{equation}
\end{definition}

Порядок і похибка апроксимації багатошарових різницевих схем визначається звичайним чином. \medskip

У [??] доводиться, що істинні такі твердження:
\begin{proposition}
    Якщо різницева схема стійка і її похибка апроксимації прямує до нуля при $\Delta t \to 0$, то умова узгодженості для цієї різницевої схеми виконується.
\end{proposition}
	
\begin{proposition}
    Для коректно поставленої крайової задачі і її багатошарової апроксимації, яка задовольняє умови узгодженості, стійкість є необхідною і достатньою умовою збіжності.
\end{proposition}

\section{Узгодженість і порядок точності}

При дослідженні узгодженості і точності різницевих схем бажано різницеві рівняння подати їх у формі 
\begin{equation}
    B_q \vec u^{n + q} + B_{q - 1} \vec u^{n + q - 1} + \ldots + B_0 \vec u^n = 0,
\end{equation}
де рівняння не розв'язане відносно $\vec u^{n + q}$ Невизначеність внаслідок множення зліва на довільну не вироджену матрицю знімається тим припущенням, що для довільної достатньо гладкої функції $u(t)$ різницевий оператор в лівій частині апроксимує оператор $\frac{\partial}{\partial t} - A$, тобто
\begin{equation}
    B_q u(t + q \Delta t) + \ldots + B_0 u(t) - \left(\frac{\partial}{\partial t} - A \right) u(t) = o(1)
\end{equation}
при $\Delta t \to 0$. \medskip
 
\begin{definition}
    \textit{Порядок точності} різницевої схеми визначимо як найбільше дійсне число $\rho$, для якого 
    \begin{equation}
        \label{eq:l4.7}
        \| B_q u(t + q \Delta t) + \ldots + B_0 u(t) \| = o((\Delta t)^\rho)
    \end{equation}
    при $\Delta t \to 0$ для всіх достатньо гладких точних розв'язків диференціального рівняння $\frac{\partial u}{\partial t} = A u$.
\end{definition}

\begin{definition}
    Вираз у лівій частині \eqref{eq:l4.7} є \textit{похибкою апроксимації} багатошарової різницевої схеми.
\end{definition}

Встановимо зв'язок між узгодженістю та порядком точності різницевих схем. Для цього повернемось до рівняння \eqref{eq:l4.2} зі сталими коефіцієнтами і умови періодичності розв'язку. Розв'язок задачі виразимо через оператори $C_j = -B_q^{-1} B_j$. \medskip

Виходячи з обмеженості операторів $C_1, \ldots, C_{q - 1}$ та з того, що виконується умова \eqref{eq:l4.7}, можна показати, що $B_q^{-1} = O(\Delta t)$. \medskip

Отже у вектора $\tilde C(\Delta t) \tilde S \tilde u(t)$ відмінною від нуля є тільки перша компонента і нам потрібно показати, що при довільному $\epsilon > 0$:
\begin{equation}
    \| (C_{q - 1} E ( (q - 1) \Delta t) + \ldots + C_0 - E (q \Delta t) ) u_0(t) \| < \epsilon \Delta t
\end{equation}
для довільного елемента $u(0)$ з деякої щільної в $B$ множини $U^0$ і для всіх достатньо малих $\Delta t$. \medskip

Нехай $B_m$ --- підпростір тригонометричних поліномів степеня $m$ з гільбертового простору $B$ періодичних функцій, а $u_0$ презентує деяку функцію $u_0(\vec x)$ з $B_m$. Якщо в рівності \eqref{eq:l4.7} покласти $u(t) = t u_0$, то 
\begin{equation}
    \label{eq:l4.8}
    L u_0 = o(1), \quad \Delta t \to 0,
\end{equation}
де
\begin{equation}
    \label{eq:l4.9}
    L = \Delta t ( q B_q + (q - 1) B_{q - 1} + \ldots + B_1 ) - E.
\end{equation}

Оскільки $B_m$ має скінчений базис, кожен елемент якого задовольняє співвідношення \eqref{eq:l4.8}, то для норми оператора $L$ в просторі $B_m$ маємо $\|L\|_m < \epsilon_1$ для достатньо малих $\Delta t$, де $\epsilon_1$ --- довільне додатне число. Тому оператор $E + L$ має обернений в цьому просторі, а саме
\begin{equation*}
    (E + L)^{-1} = E + M,
\end{equation*}
де
\begin{equation*}
    \|M\|_m < \epsilon_2 = \frac{\epsilon_1}{1 - \epsilon_1}.    
\end{equation*}

Таким чином
\begin{equation*}
    (q B_q + \ldots + B_1)^{-1} = \Delta t (E + M).
\end{equation*}

За визначенням $C_j = -B_q^{-1} B_j$, тому
\begin{equation}
    \label{eq:l4.10}
    B_q^{-1} = (q E - (q - 1) C_{q - 1} - \ldots - C_1) \Delta t (E + M).
\end{equation}

Якщо $u_0 \in B_m$, а $A$ --- диференціальний оператор рівняння 
\begin{equation}
    \frac{\partial u}{\partial t} = A u
\end{equation}
із сталими коефіцієнтами, то $u(t) \in B_m$. Крім того, для цього розв'язку буде $\frac{\partial u}{\partial t} - A u = 0$, оскільки в силу \eqref{eq:l4.7} 
\begin{equation}
    \label{eq:l4.11}
    \| (B_q E (q \Delta t) + B_{q - 1} E ((q - 1) \Delta t) + \ldots + B_0) u(t) \| < \epsilon_0
\end{equation}
для довільного $\epsilon_0 > 0$ та для всіх достатньо малих $\Delta t$. \medskip

Помноживши \eqref{eq:l4.10} на \eqref{eq:l4.11}, одержимо
\begin{equation}
    \label{eq:l4.12}
    \| ( E (q \Delta t) + C_{q - 1} E ((q - 1) \Delta t) + \ldots + C_0 ) u(t) \| < \epsilon_3 \Delta t,
\end{equation}
де 
\begin{equation*}
    \epsilon_3 = \epsilon_0 \|I + M\|_m \max_{0 < \Delta t < \tau} \| q E - (q - 1) C_{q - 1} - \ldots - C_0 \|.
\end{equation*}

Максимум, який входить в останню рівність існує, оскільки ми припустили, що різницеві рівняння стійкі. Звідси випливає, що оператори $C_{q - 1}(\Delta t), \ldots, C_1(\Delta t)$  рівномірно обмежені при $0 < \Delta t < \tau$. \medskip

Нерівність вигляду \eqref{eq:l4.12} має місце для довільного розв'язку, який належить $B_m$. Якщо розглянути всі можливі підпростори $B_m$ з простору $B$, то можливі початкові елементи $u(0)$ для цих розв'язків утворять щільну множину в $B$. \medskip

Отже, доведено: 
\begin{theorem}
    Якщо різницева схема стійка і похибка апроксимації прямує до нуля при $\Delta t \to 0$, то умова узгодженості \eqref{eq:l4.4} виконується. 
\end{theorem}

\section{Завдання для самостійної роботи}

\shortHomeworkDescription{Побудувати та дослідити тришарові явні та неявні різницеві схеми для рівнянь параболічного та гіперболічного типів другого порядку}

\chapter{Дослідженні стійкості чисельних алгоритмів із змінними операторними коефіцієнтами}

\shortLectureDescription{Дослідження стійкості чисельних алгоритмів із змінними операторними коефіцієнтами. Енергетичний метод.}

\section{Огляд інших умов стійкості нелінійних задач і задач із змінними коефіцієнтами}

\textbf{1.} Нехай розв'язком задачі є вектор функція $\vec u(\vec x, t)$, яка має неперервні похідні по $\vec x$ до порядку $p$ включно. Визначаємо норму $\|\vec u\|_p$ за рівністю
\begin{equation*}
    \|\vec u\|_p^2 = \sum_{|\vec v| \le p} \left\| \partial_{x_1}^{v_1} \partial_{x_2}^{v_2} \dots \partial_{x_d}^{v_d} \vec u\right\|^2,
\end{equation*}
де $|\vec v| = v_1 + v_2 + \dots + v_d$ означає порядок похідної, середньоквадратична норма якої стоїть під знаком суми. Задачу вважатимемо коректно поставленою, якщо її розв'язок і початковий елемент задовольняють умову
\begin{equation*}
    \|\vec u(\vec x, t)\| \le \const \|\vec u(\vec x, 0)\|_p, \quad \forall t \in [0, T]
\end{equation*}
для деякого фіксованого $p$ і для щільної множини початкових елементів. Це відповідає визначенню неперервної залежності порядку $p$ від початкових даних. \medskip

\textbf{2.} Умови \textit{сильної стійкості} визначаються так: 
\begin{enumerate}
    \item при кожному фіксованому $\Delta t$ оператор $H(\Delta t)$ визначений всюди і такий, що якщо позначити 
    \begin{equation*}
        \|\vec u\|_H^2 = (\vec u, H \vec u),
    \end{equation*}
    то існує додатна стала $K_1$, незалежна від параметрів різницевої схеми, така що 
    \begin{equation*}
        K_1^{-1} \|\vec u\|^2 \le \|\vec u\|_H^2 \le K_1 \|\vec u\|^2;
    \end{equation*}
    \item розв'язок задачі задовольняє умові
    \begin{equation}
        \label{eq:7.1}
        \|\vec u^{n+1}\|_H^2 \le (1 + K_2 \Delta t) \|\vec u^n\|_H^2,
    \end{equation}
    де $K_1$ і $K_2$ обмежені додатні сталі.
\end{enumerate}

\begin{remark}
    Це визначення стійкості в загальному випадку принаймні не слабкіше вище уведеного, бо якщо різницева схема сильно стійка, то при $n \Delta t \le T$:
    \begin{equation*}
        \|\vec u^n\| \le K_1^{1/2} \|\vec u^n\|_H \le (1 + K_2 \Delta t)^n K_1^{1/2} \|\vec u^0\|_H \le K_1 e^{K_2 T} \|\vec u^0\|_H.
    \end{equation*}
    Отже для операторів зі сталими коефіцієнтами ці визначення еквівалентні.
\end{remark}

Якщо побудована норма, в якій розв'язок задачі задовольняє умові \eqref{eq:7.1}, то при достатній гладкості її можна використати і для різницевих задач. \medskip

\textbf{3.} Якщо 
\begin{equation*}
    G(x, \xi) = \sum_\beta c^\beta e^{I \beta \xi} = 1 - a(x) \xi + O(\xi^2)
\end{equation*} 
--- локальний множник переходу різницевої схеми 
\begin{equation*}
    u^{n + 1}(x) = (C u^n(x)) \equiv \sum_\beta c^\beta u^n (x + \beta \Delta x)
\end{equation*}
для рівняння теплопровідності 
\begin{equation*}
    \frac{\partial u}{\partial t} = a(x) \frac{\partial^2 u}{\partial x^2}
\end{equation*}
із змінним коефіцієнтом $a(x) \ge \alpha > 0$, то, як встановлено роботі [5], різницева схема стійка при виконанні умов: 
\begin{enumerate}
    \item $a(x)$ задовольняє умові Ліпшиця,
    \item $\forall x$, $\forall \xi \in [-\pi, \pi] \setminus \{0\}$: $|G(x, \xi)| < 1$,
    \item $G(x, 0) = 1$ (умова узгодженості).
\end{enumerate}

\textbf{4.} При розгляді гіперболічних рівнянь, різницеву схему 
\begin{equation}
    \label{eq:7.2}
    \vec u^{n + 1} = C(\Delta T) \vec u^n
\end{equation}
визначимо як \textit{дисипативну порядку $2r$} (де $r$ натуральне число), якщо існує таке $\delta > 0$, що для всіх $x$ і $\xi$, які задовольняють умові $\max|\xi| \le \pi$  та $\Delta t$ менших деякого $\tau > 0$ виконується
\begin{equation*}
    |\lambda_v(x, \Delta t, \xi)| \le 1 - \delta |\xi|^{2r},
\end{equation*}
де $\lambda_v$ --- власні значення локальної матриці переходу $G$.

\begin{theorem}
    [13] Якщо різницева схема \eqref{eq:7.2} апроксимує диференціальне рівняння 
    \begin{equation*}
        \frac{\partial \vec u}{\partial t} = \sum_{j = 1}^d A_j(\vec x) \frac{\partial \vec u}{\partial x_j}; \quad x_j \in \RR; \quad t \in [0, 1],
    \end{equation*}
    де $\vec u = \vec u(\vec x, t)$ --- вектор розмірності $p$, $A_j(\vec x)$ --- $(p \times p)$-ермітові матриці гладко залежні від $\vec x$, але незалежні від $t$, рівномірно обмежені та рівномірно задовольняють умові Ліпшиця по $\vec x$, схема \eqref{eq:7.2} дисипативна порядку $2r$ і має порядок точності $2r - 1$, то вона стійка.
\end{theorem}

\textbf{5.} Нехай $\delta x_j = q_j(\Delta t)$, тоді різницева схема 
\begin{equation}
    \label{eq:7.3}
    \vec u^{n + 1} = C(\Delta t) \vec u^n = \left[ \sum_\beta c^\beta(\vec x) T^\beta \right] \vec u^n.
\end{equation}
стійка, якщо [13]:
\begin{enumerate}
    \item матриці $c^\beta(\vec x)$ ермітові, не залежать від $\Delta t$ і задовольняють умові Ліпшиця по компонентам вектора $\vec x$; 
    \item $c^\beta(\vec x) \ge 0$. 
\end{enumerate}

\begin{theorem}
    Якщо $c^\beta(\vec x)$ --- дійсні симетричні матриці, їх коефіцієнти не залежать від $\Delta t$ і мають обмежені другі похідні по $\vec x$, а умова 
    \begin{equation}
        \label{eq:7.4}
        \|G(\vec x, \xi)\| = \left\| \sum_\beta c^\beta(\vec x) e^{I \beta \xi} \right\| \le 1 \quad \forall x, \xi
    \end{equation}
    виконується всюди, то для дійсних векторних функцій $\vec u^n$ різницева схема \eqref{eq:7.3} стійка [13].
\end{theorem}

В роботах Лакса та Ніренберга встановлено умову, яка визначає прямий зв'язок між локальною стійкістю \eqref{eq:7.4} та стійкістю в цілому.

\begin{theorem}
    Нехай дійсні симетричні матричні коефіцієнти $c^\beta(\vec x)$ не залежать від $\Delta t$ і мають обмежені другі похідні по $\vec x$, а умова \eqref{eq:7.4} виконується всюди. Тоді для дійсних векторних функцій $\vec u^n$ різницева схема \eqref{eq:7.3} стійка, тобто 
    \begin{equation*}
        \|C(\Delta t)\| \le 1 + O(\Delta t).
    \end{equation*}
\end{theorem}

\textbf{6.} Для нелінійних задач до сьогодні ще не створено ефективних методів дослідження достатньо широких класів різницевих схем. Мало досліджені і різницеві схеми, які апроксимують розривні розв'язки. Тут найбільш розповсюджені методи лінеаризації та використання першої варіації розв'язку [5].  \medskip

Так, нехай маємо задачу Коші для системи гіперболічних рівнянь 
\begin{equation}
    \label{eq:7.5}
    \frac{\partial \vec u}{\partial t} = \sum_{j = 1}^d A_j(\vec u, \vec x) \frac{\partial \vec u}{\partial x_j}; \quad x_j \in \RR; \quad t \in [0, 1]
\end{equation}
з початковими умовами 
\begin{equation*}
    \vec u(\vec x, 0) = \vec u_0(\vec x).
\end{equation*}

Різницеву схему, яка апроксимує цю задачу, запишемо у вигляді
\begin{equation}
    \label{eq:7.6}
    \vec u^{n + 1} = \vec \phi(\vec u^n, \vec x, \Delta t),
\end{equation}
де права частина є нелінійною функцією від $\vec u^n$ та $\vec u^{n + 1}$, заданих в скінченому числі точок шаблону оператора. Нехай оператор $\vec \phi$ з рівняння \eqref{eq:7.6} є погодженим в розумінні апроксимації і його перша варіація стійка в $L_2$. \medskip

Тоді внаслідок узгодженості для будь-якої гладкої функції $\vec v$:
\begin{equation*}
    \vec \phi(\vec v, \vec x, \Delta t) = \vec v + \Delta t \sum_{j = 1}^d A_j(\vec v, \vec x) \frac{\partial \vec v}{\partial x_j} + o(\Delta t).
\end{equation*}

Нехай
\begin{equation*}
    \vec U(\vec x, t, \Delta t) = \vec u(\vec x, t) + \sum_m (\Delta t)^m \vec U_m (\vec x, t)
\end{equation*}
задовольняє \eqref{eq:7.6}. \medskip

Тоді $\vec U_m(\vec x, t)$  задовольняють лінійним рівнянням
\begin{equation}
    \label{eq:7.7}
    \frac{\partial \vec U_m}{\partial t} = \sum_{j = 1}^d A_j(\vec u, \vec x) \frac{\partial U_m}{\partial x_j} + \alpha(\vec x, t) \vec U_m + \beta_m(\vec x, t), \quad \vec U_m(\vec x, 0) = 0, \quad m = 0, 1, 2, \dots
\end{equation}

Тут $\alpha$ залежить від перших похідних $\vec u$ та перших похідних $A_j$ по $\vec u$. Розв'язки \eqref{eq:7.7} існують і неперервно залежать від неоднорідних членів $\beta_m$ та їх похідних до деякого порядку $r_0$. Якщо, крім того, $\vec u$, $A_j$, $\vec \phi$ мають неперервні похідні до порядку $[\frac{d + 1}{2}] + r_0 + 2$, то при $\Delta t \to 0$ різницева апроксимація $\vec u^{n + 1}$ збігається до розв'язку задачі \eqref{eq:7.5} [5]. \medskip

Для багатьох задач жодна з цих умов не виконується і виникає необхідність шукати більш досконалі критерії стійкості. Аналіз інших відомих методів можна знайти в монографіях [1,7--9] та ін.

\section{Дослідження стійкості різницевих схем за допомогою енергетичного методу}

\subsection{Аналіз стійкості одного різницевого рівняння}

Проілюструємо ідею методу на прикладі розв'язку задачі Коші для модельного рівняння конвективного переносу [12]
\begin{equation}
    \label{eq:10.1}
    \frac{\partial u}{\partial t} = -k \frac{\partial u}{\partial x}, \quad k = \const, \quad x \in \RR, \quad t \in \RR_+,
\end{equation}
який задовольняє початковій умові
\begin{equation*}
    u(x) = \phi(x), \quad x \in \RR.
\end{equation*}

Будемо вважати, що функція $\phi(x)$ є \textit{фінітною}, тобто вона перетворюється в нуль поза межами деякого скінченого проміжку: $\phi(x) = 0$ при $|x| \ge \ell$. Тоді розв'язок задачі $u(x, t)$ в довільний момент часу $t_1$ також перетворюється в нуль при достатньо великому значенні $|x| \ge \ell(t_1)$. На сітковій множині
\begin{equation*}
    \bar \omega_{h, \tau} = \{x_i, t_n | x_i = i h, i = 0, 1, 2, \dots; t_n = n \tau, n = 0, 1, 2, \dots\}
\end{equation*}
зі сталими невід'ємними кроками сітки $h$ і $\tau$ функції $u(x, t)$ буде відповідати сіткова функція $u_i^n = u(ih, n \tau)$, причому $u_i^n = 0$ при $|i| \ge N_n$. У такому разі для фінітних функцій скалярний добуток 
\begin{equation*}
    (u, v) = \sum_{i = -\infty}^\infty u_i^n v_i^n h
\end{equation*}
запишеться як
\begin{equation*}
    (u, v) = \sum_{i = -N_n}^{N_n} u_i^n v_i^n h.
\end{equation*}

У просторі сіткових функцій уведемо різницеві оператори
\begin{equation*}
    L_h u_i^k = k \frac{u_i^n - u_{i-1}^n}{h}, \quad \text{та} \quad L_\tau u_i^k = \frac{u_i^{n+1} - u_i^n}{\tau}, \quad i = 0, \pm 1, \pm 2, \dots, \quad n = 0, 1, 2, 3, \dots
\end{equation*}
кожен з яких є також сітковою функцією того ж простору. \medskip

Використовуючи формулу підсумування за частинами [3, стор.~98], легко переконатися, що на скінченому інтервалі $|i| < N$ вірна формула
\begin{equation*}
    \sum_{i = -N - \ell}^{N + \ell} v_i^n \frac{y_i^n - y_{i-1}^n}{h} h = y_{N + \ell}^n v_{N + \ell}^n - y_{-N-1-\ell}^n v_{-N-1-\ell}^n - \sum_{i = -N-\ell-1}^{N+\ell+1} y_{i-1}^n \frac{v_{i+1}^n - v_i^n}{h} h.
\end{equation*}

Звідки, враховуючи фінітність функцій $y_i^n$ та $u_i^n$, маємо
\begin{equation*}
    \sum_{i = -N - \ell}^{N + \ell} v_i^n \frac{y_i^n - y_{i-1}^n}{h} h = -\sum_{i = -N-\ell-1}^{N+\ell+1} y_{i-1}^n \frac{v_{i+1}^n - v_i^n}{h} h.
\end{equation*}

Виходячи з визначення спряженого оператора
\begin{equation*}
    (L_h u, v) = (u, L_h^\star v)
\end{equation*}
встановлюємо, що $L_h^\star u = - k \frac{u_{i+1}^n - u_i^n}{h}$. Причому $L_h^\star \ne L_h$.

Запишемо ще одну очевидну рівність
\begin{equation*}
    y_i^n \frac{y_i^n - y_{i-1}^n}{h} = \frac{(y_i^n)^2 - (y_{i-1}^n)^2}{2h} + \frac{(y_i^n - y_{i-1}^n)^2}{2h} = \half \Big[ L_h(y)^2 + h (L_h y)^2 \Big].
\end{equation*}

Звідси після підсумування та врахування фінітності функцій випливає, що
\begin{equation*}
    (L_h y, y) = \frac{h \|L_h y\|^2}{2 k}.
\end{equation*}

Різницеве рівняння
\begin{equation}
    \label{eq:10.3}
    L_\tau u + L_h u = 0
\end{equation}
помножимо на $u_i^{n+0.5}$ і підсумуємо за усіма точками просторової сітки. У результаті одержимо
\begin{equation*}
    \half L_\tau \|u^n\|^2 = -(L_h u^n, u^{n+0.5}).
\end{equation*}

Оскільки
\begin{equation*}
   u_i^{n+0.5} = u_i^n + \tau \frac{u_i^{n+0.5} - u_i^n}{\tau/2} = u_i^n - \frac{\tau}{2} L_h u_i^n,
\end{equation*}
то
\begin{equation*}
    \half L_\tau \|u\|^2 = -(L_h u^n, u^n) + \frac{\tau}{2} \|L_h u^n\|^2,
\end{equation*}
або ж
\begin{equation*}
    \half L_\tau \|u^n\|^2 = \half \left( \tau - \frac{h}{k} \right) \|L_h u^n\|^2.
\end{equation*}

Отже, якщо права частина рівності не додатна, тобто $\tau \le \frac{h}{|k|}$, то 
\begin{equation*}
    L_\tau \|u^n\|^2 \le 0
\end{equation*} 
і, отже
\begin{equation}
    \label{eq:10.5}
    \|u^{n + 1}\| \le \|u_n\|.
\end{equation}

Нерівність \eqref{eq:10.5}, яка вірна при $\tau \le \frac{h}{|k|}$, свідчить про рівномірну стійкість різницевого рівняння \eqref{eq:10.3} за початковими даними.  \medskip

Енергетичний метод встановлює достатні умови стійкості. \medskip

Відзначимо, що енергетичний метод дозволяє отримати більшу ніж в методі фон Неймана інформацію про різницеву схему. Так відомо, що центрально різницева апроксимація рівняння конвективного переносу безумовно нестійка. Тобто для довільного вибору кроків сітки модуль множника переходу більший від одиниці. Проаналізуємо тепер цю схему за допомогою енергетичного методу. Для цього уведемо оператор
\begin{equation*}
    L_h^0 u_i^n = k \frac{u_{i+1}^n - u_{i-1}^n}{2h}, \quad i = 0, \pm 1, \pm 2, \dots, \quad n = 0, 1, 2, \dots
\end{equation*}

Очевидно, що 
\begin{equation*}
    L_h^0 u_i^n = \half \Big( L_h u_i^n - L_h^\star u_i^n \Big),
\end{equation*}
а отже, в силу визначення спряженого оператора
\begin{equation*}
    (L_h^0 u, u) = \half \Big[ (L_h u, u) - (L_h^\star u, u) \Big] = 0.
\end{equation*}

Запишемо різницеве рівняння 
\begin{equation}
    \label{eq:10.6}
    L_\tau u_i^n + k L_h^0 u_i^n = 0,
\end{equation}
помножимо його на $u_i^{n+0.5}$ і підсумуємо за усіма індексами просторових точок сітки та, скориставшись апроксимуючим різницевим рівнянням, одержимо
\begin{equation*}
    \half L_\tau \|u^n\|^2 = - (L_h^0 u^n, u^{n+0.5}) = -(L_h^0 u^n, u^n) + \frac{\tau}{2} \|L_h^0 u^n\|^2.
\end{equation*}

Врахувавши, що $(L_h^0 u, u) = 0$, маємо
\begin{equation*}
    L_\tau \|u^n\|^2 = \tau \|L_h^0 u^n\|^2 \ge 0.
\end{equation*}
 
Отже, норма сіткового розв'язку при переході на наступний часовий шар не може спадати. Проте її зростання обмежене. Дійсно
\begin{align*}
    \|L_h^0 u^n\|^2
    &= k^2 \sum_i (L_h^0 u_i^n)^2 h
    = \frac{k^2}{4h^2} \sum_i (u_{i+1}^n - u_{i-1}^n)^2 h \le \\
    &\le
    \frac{k^2}{2h^2} \sum_i \Big( (u_{i+1}^n)^2 + (u_{i-1}^n)^2 \Big) h \le \frac{k^2}{h^2} \|u^n\|^2.
\end{align*}

Звідки випливає, що
\begin{equation*}
    L_\tau \|u^n\|^2 \le \frac{k^2}{h^2} \|u^n\|^2,
\end{equation*}
або
\begin{equation*}
    \|u^{n+1}\|^2 \le \left[ 1 + \frac{k^2 \tau}{h^2} \tau \right] \|u^n\|^2.
\end{equation*}
 
Нехай $d_0 \ge \frac{k^2 \tau}{h^2}$. Тоді вірна мажорантна оцінка
\begin{equation*}
    1 + \frac{k^2 \tau}{h^2} \tau  \le e^{d_0 \tau},
\end{equation*}
з якої випливає
\begin{equation*}
    \|u^{n+1}\| \le e^{0.5 d_0 \tau} \|u^n\|.
\end{equation*}

Ця нерівність приводить до умови стійкості за початковими даними такого виду
\begin{equation*}
    \|u^{n+1}\| \le e^{0.5 d_0 T} \|u^n\|.
\end{equation*}

Тут $T$ граничне значення інтервалу часу, на якому розглядається розв'язок задачі. \medskip

Відзначимо, що умова стійкості $\tau \ge \frac{d_0 h^2}{k^2}$ не є природною для рівнянь гіперболічного типу. Вона є більш притаманною для параболічних рівнянь.

\subsection{ Застосування енергетичного методу до системи різницевих рівнянь}

В області $(x, t) \in \RR \times \RR_+$ розглянемо задачу Коші для системи диференціальних рівнянь першого порядку 
\begin{equation*}
    ,
\end{equation*}
при заданих початкових умовах
\begin{equation*}
    .
\end{equation*}

Вважаємо, що функції $u$ та $v$ є фінітними. Помножимо перше рівняння системи на $v$, а друге на $u$ і складемо результати
\begin{equation*}
    .
\end{equation*}

Зінтегрувавши останню рівність по $x$ одержимо
\begin{equation*}
    ,
\end{equation*}
де $E(t) = \half \int_\RR (u^2 + v^2) \diff x$. \medskip

З останньої диференціальної рівності випливає енергетична тотожність
\begin{equation*}
    E(t) = E(0),
\end{equation*}
яка виражає закон збереження енергії. \medskip

Перейдемо далі до дослідження різницевої схеми. Диференціальній задачі  поставимо у відповідність сімейство різницевих схем
\begin{gather*}
    , \\
    ,    
\end{gather*}
де $\alpha$ та $\beta$~--- вільні параметри. \medskip

Помножимо перше рівняння системи на  , а друге на   і підсумуємо одержані співвідношення за індексами просторових точок
\begin{align*}
    &= , \\
    &= .
\end{align*}

Склавши ці рівняння, після простих перетворень [12], маємо
\begin{equation*}
    L_\tau J = Q,
\end{equation*}
або
\begin{equation*}
    J^{n + 1} = J^n + \tau Q,
\end{equation*}
де
\begin{equation*}
    \quad .
\end{equation*}

Якщо для якоїсь з конкретних схем (тобто при конкретно заданих значеннях параметрах $\alpha$ та $\beta$) може бути встановлено умову
\begin{equation*}
    Q \le 0
\end{equation*}
то ця умова є достатньою умовою стійкості у енергетичній нормі. Тобто
\begin{equation*}
    J^{n+1} \le J^n.
\end{equation*}

Так, якщо в системі різницевих рівнянь покласти $\beta = 0.5$, залишивши $\alpha$ вільним параметром, то $Q = -\tau (\alpha - 0.5) \|L_\tau u^n\|^2$, а отже для виконання нерівності $Q \le 0$, яка забезпечує стійкість схеми достатньо вимагати $\alpha \ge 0.5$. \medskip

Наприкінці відзначимо, що при $\alpha = 0.5$ апроксимація є стійкою і $Q = 0$. Остання рівність означає виконання на сітковій множині закону збереження кількості енергії, який в цьому разі має вигляд
\begin{equation*}
    \|u^{n+1}\|^2 + \|v^{n+1}\|^2 = \|u^0\|^2 + \|v^0\|^2.
\end{equation*}

Якщо ж $0.5 < \alpha \le 1$, то $Q \le 0$ і енергія в різницевій схемі спадає внаслідок дії дисипативних властивостей притаманних самій різницевій схемі. Швидкість спадання енергій різницевої схеми є величиною порядку $O(\tau)$.

\subsection{Завдання для самостійної роботи}

\shortHomeworkDescription{Принцип заморожених коефіцієнтів. Ознака Бабенка---Гельфонда}

\section{Принцип заморожених коефіцієнтів та ознака Бабенка і Гельфанда}

При дослідженні стійкості різницевих рівнянь з змінними коефіцієнтами важливе місце займає принцип заморожених коефіцієнтів [7]. \medskip

В області визначення розв'язку вибирається довільна точка $(\tilde x, \tilde t)$. Значення коефіцієнтів різницевого рівняння, визначеного на шаблоні цієї точки, фіксуються (заморожуються) і вважається, що в усіх точках області розв'язуються різницеві рівняння з цими сталими коефіцієнтами. За допомогою спектральної ознаки встановлюються умови стійкості одержаної різницевої задачі з сталими коефіцієнтами. Дослідивши побудовані, таким чином, різницеві задачі в усіх точках сіткової області, за необхідну умову стійкості задачі вибираємо найбільш жорстку умову.  \medskip

Тобто [7], має місце 

\begin{proposition}[принцип заморожених коефіцієнтів]
    Для стійкості лінійної різницевої задачі з змінними коефіцієнтами необхідно, щоб кожна задача Коші для різницевого рівняння з сталими замороженими коефіцієнтами задовольняла необхідній спектральній ознаці стійкості фон Неймана. 
\end{proposition}

Така ознака стійкості є необхідною умовою і має назву ознаки локальної стійкості. \medskip

Розглянемо, наприклад, різницеву схему
\begin{gather}
    \label{eq:6.1}
    , \\
    ,
\end{gather}
де $\ell_1$ та $\ell_2$~--- різницеві оператори граничних умов. В довільній точці сіткової області $(\tilde x, \tilde t)$ зафіксуємо значення коефіцієнта $a(x, t) = a(\tilde x, \tilde t)$ і дослідимо одержану різницеву схему зі сталим коефіцієнтом
\begin{equation}
    =
\end{equation}
у всіх внутрішніх точках сіткової області. \medskip

Як встановлено вище, умовою стійкості такого різницевого рівняння є виконання нерівності
\begin{equation*}
    \le .
\end{equation*}

Оскільки в силу принципу заморожених коефіцієнтів для стійкості різницевої задачі ця умова повинна виконуватись для всіх точок $(\tilde x, \tilde t)$ з області визначення розв'язку, то умовою локальної стійкості для рівняння з змінним коефіцієнтом є умова
\begin{equation*}
    \le .
\end{equation*}

Принцип заморожених коефіцієнтів грунтується на евристичному рівні строгості і при дослідженні нелінійних задач. Так [7], для нелінійної задачі
\begin{gather*}
    , \\
    ,
\end{gather*}
використаємо різницеву схему
\begin{gather*}
    , \\
    .
\end{gather*}

В цій різницевій схемі крок за часом може змінюватись при переході від одного часового шару до іншого. Нехай якимось чином ми дійшли до часового шару $t_p$. Виникає природне питання яким вибрати наступний крок $\tau_{p+1}$, щоб одержана різницева схема була стійкою? Прогнозувати оцінку цього параметра дозволяє принцип заморожених коефіцієнтів, виходячи з встановленої вище оцінки
\begin{equation*}
    \tau_{p+1} \le \frac{h^2}{2 \max_i |1 + (u_i^n)^2|}.
\end{equation*}

Різноманітні чисельні експерименти на ЕОМ підтверджують вірність цього евристичного твердження [7]. \medskip

Якщо необхідну умову стійкості для задачі Коші з замороженими коефіцієнтами у якійсь точці буде порушено, то не можна чекати стійкості розв'язку крайової задачі ні при яких крайових умовах. \medskip

Принцип заморожених коефіцієнтів не враховує вплив граничних умов. Локальна стійкість може мати місце при одних і не мати місця при інших граничних умовах. Однією з причин цього є локальна некоректність постановки різницевої задачі. Відомі приклади Крайса [5, 12], де в одному з них для коректно поставленої крайової задачі для рівняння з змінними коефіцієнтами, всі задачі з відповідними замороженими коефіцієнтами некоректні, а в другому~--- всі задачі зі сталими коефіцієнтами коректні, в той час як задача зі змінними коефіцієнтами такою не є. \medskip

Стренг довів [5], що для коректно поставленої задачі для рівняння $\frac{\partial u}{\partial t} = A u$, після заміни оператора $A$ його головною частиною (тобто сумою членів, які містять похідні найвищого порядку) всі задачі з сталими коефіцієнтами є коректно поставленими. Тому всі задачі для систем рівнянь першого порядку є локально коректні. Для симетричних гіперболічних систем вищих порядків коректність постановки лінійних задач із змінними коефіцієнтами забезпечується теоремами чисто локального характеру. \medskip

При дослідженні стійкості різницевих початково–крайових задач з змінними коефіцієнтами і врахуванні впливу можливих збурень в крайових умовах використовується ознака Бабенка---Гельфанда. Вона вимагає одночасного дослідження трьох різницевих задач: 
\begin{enumerate}
    \item задачі Коші з збуренням в точці, яка лежить всередині області
    \begin{equation*}
        =
    \end{equation*}
    без урахування граничних умов (задача Коші); 

	\item дослідження різницевої задачі у всіх точках справа від лівої граничної точки $x = x_0$ з врахуванням тільки граничної умови в цій точці з замороженими в ній коефіцієнтом
    \begin{gather*}
        , \\
        ,
    \end{gather*}
    віднісши при цьому праву границю в нескінченно віддалену точку (це можливо тільки для тих розв'язків, для яких $u_i^n \to 0$ при $i \to \infty$);

	\item  розглянути задачу з врахуванням тільки правої граничної умови, з замороженими в ній коефіцієнтами і дослідити її стійкості в області зліва від даної точки границі
    \begin{gather*}
        , \\
        .
    \end{gather*}
    віднісши ліву границю в $-\infty$ (це можливо тільки для тих розв'язків, для яких $u_i^n \to 0$ при $i \to -\infty$).
\end{enumerate}

Останні дві задачі є значно простішими від першої задачі, оскільки при фіксованому відношенні кроків $r = \frac{\tau}{h^2}$ вони не залежать від $h$ і є задачами з сталими коефіцієнтами. \medskip

Отже, потрібно скласти три незалежні від $h$ різницеві задачі і для кожної з (попередньо поклавши $r = \frac{\tau}{h^2}$) знайти усі ті власні числа оператора переходу (з часового шару $n$ на шар $n+1$) при яких існує обмежений при прямуванні $x$ до $+ \infty$ чи $- \infty$ розв'язок виду
\begin{equation}
    \label{eq:6.2}
    u_i^n = \lambda^n u_i^0
\end{equation}
при виконанні вказаних у цих задачах граничних умов.

\begin{proposition}
    Для стійкості початково–крайової задачі з змінними коефіцієнтами спектри операторів переходу кожної з трьох наведених задач повинні лежати в крузі одиничного радіуса.
\end{proposition}

Методику визначення відповідних власних чисел  $\lambda$ оператора переходу проілюструємо модельному прикладі. Для спрощення викладок покладемо   і обчислимо спектри всіх трьох задач при різних крайових умовах   та  .
	Після підстановки розв'язку задачі у вигляді (6.2) у різницеве рівняння, (6.1), одержимо
 ,    ,
або
 .
	Характеристичним рівнянням цього різницевого рівняння є квадратне рівняння [5]
	 .	(6.3)
	Якщо   є коренем рівняння (6.3), то сіткова функція 
 
є одним з розв'язків різницевого рівняння
	 .	(6.4)
	При  , тобто   ( ), обмеженим при   та   розв'язком є сіткова функція [5]
	 .	(6.5)
Після підстановки (6.5) у (6.4), мамо
 ,
або, врахувавши, що  , знаходимо
 ,   .
Ці значення   заповнюють відрізок  , який є спектром розглянутої задачі. Поза цим відрізком власних чисел не існує, оскільки у випадку відсутності у характеристичного рівняння кореня   рівного за модулем одиниці різницева задача не має обмежених при   розв'язків.
	Якщо   не лежить на відрізку  , то модулі обох коренів характеристичного рівняння відмінні від одиниці, але їх добуток рівний вільному члену, тобто одиниці. 
	Нехай для визначеності  . а  .Тоді загальний розв'язок різницевого рівняння, модуль якого спадає при  , має вигляд
 ,
а загальний розв'язок різницевого, яке прямує до нуля при  , має вигляд
 .
	Для визначення власних значень другої різницевої задачі потрібно підставити   у ліве граничне значення   і знайти ті значення   при яких воно виконується. Тобто, множина Це будуть усі власні значення другої різницевої задачі. 
	Так при  , умова   не виконується ні при яких  , отже власних значень на границі не існує і множина  ,   містить усі власні числа.
	Умова   приводить до рівняння
 .
Оскільки  , то при   власних чисел немає.
	Якщо ж, наприклад,  , то умова 
 
 виконується при  . Використовуючи рівняння
 ,
знаходимо
 .
Це і є єдине власне значення різницевої задачі. Воно лежить поза одиничним кругом.
	Аналогічно, використовуючи граничну умову
  при  ,
знаходимо власні значення третьої різницевої задачі.

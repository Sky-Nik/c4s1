\subsection{Метод Рітца}
% нагадування попередньої лекції
\begin{theorem}
    якась теорема з двома пунктами
\end{theorem}
\begin{proof}
    якесь доведення
\end{proof}

\subsubsection{Мінімізаційна послідовність}

\begin{enumerate}
    \item $\{\phi_i\} \in D(A)$ --- повна;
    \item $H_n = \mathcal{L}(\phi_1, \ldots, \phi_n)$;
    \item $u_n = \sum_{i = 1}^n c_i \phi_i$;
\end{enumerate}

\begin{gather*}
    \inf_{u \in H_n} \Phi(u) = \Phi(u_n) \\
    G(u_n, v) = (f, v), \quad \forall v \in H_n \\
    G(u_n, \phi_j) = (f, \phi_j), \quad j = \overline{1, n} \\ 
    \sum_{i = 1}^n c_i G(\phi_i, \phi_j) = (f, \phi_j), \quad j = \overline{1, n} \\
    \sum_{i = 1}^n c_i (A \phi_i, \phi_j) = (f, \phi_j), \quad j = \overline{1, n}.
\end{gather*}

\begin{theorem}
    \begin{enumerate}
        \item Якщо $u^\star$ --- розв'язок \eqref{eq:3.1}, а оператор $A$ задовольняє умовам \eqref{eq:3.3} і \eqref{eq:3.4}, то мінімум функціонала Рітца \eqref{eq:3.5} буде досягатися на $u^\star$, причому тільки на ньому.
        \item Якщо то мінімум функціонала Рітца \eqref{eq:3.5} досягається на елементі $u^\star \in D(A)$, то $u^\star$ --- розв'язок \eqref{eq:3.1}.
    \end{enumerate}
\end{theorem}

\begin{proof}
    \begin{enumerate}
        \item Якщо $u^\star$ --- розв'язок \eqref{eq:3.1}, то
        \begin{equation}
            \label{eq:3.10}
            \Phi(u) = (Au, u) - 2 (Au^\star, u) = (Au, u) - (Au^\star, u) + (Au^\star, u^\star) - (Au^\star, u) - (A u^\star, u^\star) = (A (u - u^\star), (u - u^\star)) - (A u^\star, u^\star).
        \end{equation}
        Оскільки $A > 0$ то з \eqref{eq:3.10} маємо
        \begin{equation}
            \label{eq:3.11}
            \Phi(u) \ge \Phi(u^\star),
        \end{equation}
        оскільки 
        \begin{equation*}
            (A (u - u^\star), (u - u^\star)) \ge \mu \|u - u^\star\|^2,
        \end{equation*}
        причому рівність досягається лише коли $u = u^\star$.

        \item Нехай на $u^\star \in D(A)$ досягається мінімум функціонала \eqref{eq:3.5}, тоді
        \begin{equation}
            \label{eq:3.12}
            (A u^\star, v) = (f, v), \quad \forall v \in D(G),
        \end{equation}
        \stepcounter{equation}{1}
        аьо
        \begin{equation*}
            (A u^\star - f, v) = 0, \quad \forall v \in D(G),
        \end{equation*}
        а, оскільки $D(G)$ щільна в $H$, то
        \begin{equation}
            \label{eq:3.14}
            A u^\star = f.
        \end{equation}
    \end{enumerate}
\end{proof}

\begin{remark}
    Мінімізаційна послідовність звелася до тієї ж системи що й у методі Бубнова-Гальоркіна, то які ж його переваги?
\end{remark}

\subsubsection{Метод Рітца в $H_A$}

Справа в тому, що якщо оператор $A$ задовольняє умовам \eqref{eq:3.3} і \eqref{eq:3.4}, то можна ввести енергетичний простір $H_A$ у якому скалярний добуток введено за формулою
\begin{equation}
    \label{eq:3.15}
    (u, v)_A = (A u, u),
\end{equation}
а норма за формулою
\begin{equation}
    \label{eq:3.16}
    \|u\|_A^2 = (u, u)_A.
\end{equation}

Можна перевірити, що ці функції задовольняють аксіомам скалярного добутку та норми. Таким чином $H_A$ --- гільбертовий простір, ширший за $D(A)$. \medskip

Тоді задачі \eqref{eq:3.1} ставимо у відповідність задачу
\begin{equation}
    \label{eq:3.18}
    \inf_{u \in H_A} \Phi(u) = \Phi(u_n),
\end{equation}
для якої мають місце аналогічні результати:

\begin{theorem}
    \begin{enumerate}
        \item \begin{equation}
            \label{eq:3.19}
            \Phi(u) = (A u, u) - 2 (f, u) = \|u - u^\star\|_A^2 - \|u^\star\|_A^2,
        \end{equation}
        \item \begin{equation}
            \label{eq:3.20}
            \inf_{u \in H_n \subset H_A} \Phi(u) = \Phi(u_n) = \Phi_0.
        \end{equation}
    \end{enumerate}
\end{theorem}

З \eqref{eq:3.20} бачимо
\begin{equation*}
    (A u_n, v) = (f, v), \quad v \in H_n,
\end{equation*}
тоді
\begin{equation*}
    (A u_n - A u^\star, v) = 0, \quad v \in H_n,
\end{equation*}
або
\begin{equation}
    \label{eq:3.21}
    (u_n - u^\star, v)_A = 0, \quad v \in H_n.
\end{equation}

Це означає, що
\begin{equation}
    \label{eq:3.22}
    u = P_{H_n} u^\star
\end{equation}

\begin{theorem}
    Нехай координатна система $\{\phi_i\}$ є повною в $H_A$, тоді при $n \to \infty$ мінімізуюча послідовність $\{u_n\}$ метода Рітца збігається до розв'язку задачі \eqref{eq:3.1} в нормі простору $H_A$.
\end{theorem}

\begin{example}
    Задано рівняння
    \begin{equation*}
        -(k u')' + qu = f, \quad 0 < x < 1
    \end{equation*}
    з крайовими умовами
    \begin{equation*}
        u(0) = u(1) = 0,
    \end{equation*}
    і функціями
    \begin{equation*}
        k(x) \ge c_0 > 0, \quad q(x) \ge 0.
    \end{equation*}
\end{example}

\begin{remark}
    Взагалі кажучи, перед тим як будь-що робити, ми маємо показати симетричність $A$:
    \begin{equation}
        \label{eq:3.23}
        \forall v \in D(A): \quad (A u, v) = \int_0^1 \left( -(ku')' v + q u v) \diff x = - \left. k u' v\right|_0^1 + \int_0^1 \left( -k u' v' + q u v) \diff x = (u, A v).
    \end{equation}
    а також додатну визначеність: з $u(0) = 0$ маємо
    \begin{equation*}
        u(x) = \int_0^x u'(\xi) \diff \xi,
    \end{equation*}
    звідки
    \begin{equation}
        \label{eq:3.24}
        \int_0^1 u^2(x) \diff x = \int_0^1 \left( \int_0^x u'(\xi) \diff \xi \right)^2 \diff x \le \int_0^1 \int_0^x \xi \diff \ix \int_0^x (u'(\xi))^2 \diff \xi \diff x \le \frac{1}{2} \|u'\|^2,
    \end{equation}
    або ж 
    \begin{equation}
        \label{eq:3.25}
        \|u\|^2 \le \frac{1}{2} \|u'\|^2, \quad u \in D(A)
    \end{equation}
    \stepcounter{equation}

    Тоді, підставляючи в \eqref{eq:3.23} $u$ замість $v$ маємо
    \begin{equation}
        \label{eq:3.27}
        (A u, г) = \int_0^1 \left( -(ku')' г + q u^2) \diff x \ge c_0 \int_0^1 (u'(x))^2 \diff x + \int_0^1 q (u(x))^2 \ge c_0 \int_0^1 (u'(x))^2 \diff x \ge c_1 \|u\|^2,
    \end{equation}
    де $c_1 = 2 c_0$. Розгялдаючи ліву і праву частини цієї рівності маємо додатновизначеність $A$.
\end{remark}

\begin{solution}
    \begin{enumerate}
        \item Класичний розв'язок: функціонал:
        \begin{equation*}
            \Phi(u) = \int_0^1 (-(ku')' u + q u^2 - 2 f(u)) \diff x,
        \end{equation*}
        з множиною
        \begin{equation*}
            F(A) = \{ u \in C^2([0, 1]), u(0) = u(1) = 0\}.
        \end{equation*}

        У якості $\{\phi_i\}$ можна взяти $\phi_i(x) = x^i (1 - x)$, $i = \overline{1, n}$. Тоді
        \begin{equation*}
            \sum_{i = 1}^n c_i \left( \int_0^1 (k \phi_i')' \phi_j + q\phi_i \phi_j \right) \diff x = (f, \phi_j), \quad j = \overline{1, n}.
        \end{equation*}
        \item Узагальнений розв'язок: функціонал:
        \begin{equation*}
            \Phi(u) = \int_0^1 (k(u')^2 u + q u^2 - 2 f(u)) \diff x,
        \end{equation*}

        Енергетичний простір $H_A$ зі скалярним добутком
        \begin{equation}
            (u, v)_A = \int_0^1 (k u' v' + q u v) \diff x,
        \end{equation}
        елементи якого
        \begin{equation*}
            H_A = \{ u: \|u\|_A < \infty, u(0) = u(1) = 0\}.
        \end{equation*}

        Це по суті є $\overset{\circ}{W}_2^1((0, 1))$ (2 --- інтегровні з квадратом, 1 --- до першої похідної, $\circ$ --- нуль на краях). \medskip

        Приклад $\{\phi_i\}_{i = 1}^{n - 1}$ --- так звані штафетини:
        \begin{equation}
            \phi_i(x) = \begin{cases}
                \frac{x - x_{i - 1}}{h}, & x_{i - 1} \le x \le x_i, \\
                \frac{x_{i + 1} - x_i}}{h}, & x_i \le x \le x_{i + 1}, \\
                0, & x \not\in [x_{i - 1}, x_{i + 1}].
        \end{equation}
        % вставить опять картинку. на всякий случай.
    \end{enumerate}
\end{solution}

Це все, що стосується першої граничної задачі. Тепер про задачу третього роду:
\begin{example}
    Задано рівняння
    \begin{equation*}
        -(k u')' + qu = f, \quad 0 < x < 1
    \end{equation*}
    з крайовими умовами
    \begin{align*}
        - k u' + \alpha_1 u &= \beta_1, \quad x = 0, \\
        - k u' + \alpha_1 u &= \beta_1, \quad x = 1.
    \end{align*}
\end{example}

\begin{solution}
    Розглянемо тут узагальнений розв'язок
    \begin{equation*}
        \Phi(u) = \int_0^1 (k(u')^2 u + q u^2 - 2 f(u)) \diff x + \alpha_1 u^2(0) + \alpha_2 u^2(1) - 2 \beta_1 u(0) - 2 \beta_2 u(1),
    \end{equation*}

    Енергетичний простір $H_A$ з нормою
    \begin{equation}
        \|u\| = \int_0^1 (k (u')^2 + q u^2) \diff x + \alpha_1 u^2(0) + \alpha_2 u^2(1),
    \end{equation}
    елементи якого
    \begin{equation*}
        H_A = \{ u: \|u\|_A < \infty\}.
    \end{equation*}

    Приклад $\{\phi\}$ --- $\{\phi_i\}_{i = 0}^{n}$.
\end{solution}

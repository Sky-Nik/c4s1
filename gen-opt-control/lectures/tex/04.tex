\subsection{Перетворення Лапласа}

\subsubsection{Допоміжні твердження про перетворення Лапласа}

\begin{definition}
    Нехай $f: \RR_+ \to \RR$, тоді 
    \begin{equation}
        \Ltrans{f}(\eta) = \ol{f}(\eta) = \int_0^\infty e^{-\eta t} f(t) \diff t.
    \end{equation}
\end{definition}

\begin{lemma}[перетворення Лапласа похідної]
    \begin{equation}
        \Ltrans{f'}(\eta) = \eta \cdot \Ltrans{f}(\eta) - f(0).
    \end{equation}
\end{lemma}
\begin{proof}
    Інтегруємо частинами.
\end{proof}

\begin{lemma}[перетворення Лапласа згортки]
    \begin{equation}
        \Ltrans{f \star g}(\eta) = \Ltrans{f}(\eta) \cdot \Ltrans{g}(\eta).
    \end{equation}
\end{lemma}
\begin{proof}
    Змінюємо порядок інтегрування.
\end{proof}

\begin{lemma}[перетворення Лапласа степеневої функції]
    \begin{equation}
        \Ltrans{t^{-\beta}}(\eta) = \Gamma(1 - \beta) \cdot \eta^{\beta - 1}.
    \end{equation}
\end{lemma}
\begin{proof}
    За означенням
    \begin{equation}
        \Ltrans{t^{-\beta}}(\eta) = \int_0^\infty e^{-\eta t} t^{-\beta}.
    \end{equation}
    Зробимо заміну змінних: $\eta t = \xi$, $\diff t = \diff \xi / \eta$. Тоді
    \begin{equation}
        \begin{aligned}
            \int_0^\infty e^{-\eta t} t^{-\beta} &= \int_0^\infty e^{-\xi} \left(\frac{\xi}{\eta}\right)^{-\beta} \frac{1}{\eta} \diff \xi = \\
            &= \eta^{\beta - 1} \int_0^\infty e^{-\xi} \xi^{-\beta} \diff \xi = \\
            &= \eta^{\beta - 1} \Gamma(1 - \beta).
        \end{aligned}
    \end{equation}
\end{proof}

\subsubsection{Перетворення Лапласа дробового інтеграла і похідної}

\begin{lemma}[перетворення Лапласа інтеграла дробового порядку]
    \begin{equation}
        \Ltrans{\Fint{\alpha} f}(\eta) = \eta^{-\alpha} \Ltrans{f}(\eta).
    \end{equation}
\end{lemma}
\begin{proof}
    \begin{equation}
        \begin{aligned}
            \Ltrans{\Fint{\alpha} f}(\eta) &= \Ltrans{f \star y_\alpha}(\eta) = \\
            &= \Ltrans{f}(\eta) \cdot \Ltrans{y_\alpha}(\eta) = \\
            &= \Ltrans{f}(\eta) \cdot \frac{1}{\Gamma(\alpha)} \cdot \Gamma(1 - (1 - \alpha)) \cdot \eta^{-\alpha} = \\
            &= \eta^{-\alpha} \Ltrans{f}(\eta).
        \end{aligned}
    \end{equation}
\end{proof}

\begin{lemma}[перетворення Лапласа похідної Рімана-Ліувілля]
    \begin{equation}
        \Ltrans{\RLFdiff{\alpha} f}(\eta) = \eta^{\alpha} \Ltrans{f}(\eta) - \sum_{k = 0}^{n - 1} (\RLFdiff{\alpha - k - 1} f) (0) \cdot \eta^k.
    \end{equation}
\end{lemma}
\begin{example}
    Зокрема, при $0 < \alpha < 1$ маємо
    \begin{equation}
        \Ltrans{\RLFdiff{\alpha} f}(\eta) = \eta^{\alpha} \Ltrans{f}(\eta) - (\Fint{\alpha - 1} f) (0).
    \end{equation}
\end{example}
\begin{proof}
    \begin{equation}
        \begin{aligned}
            \Ltrans{\RLFdiff{\alpha} f}(\eta) &= \Ltrans{\frac{\diff}{\diff t} \Fint{1 - \alpha} f}(\eta) = \\
            &= \eta \cdot \Ltrans{\Fint{1 - \alpha} f}(\eta) - (\Fint{1 - \alpha} f)(0) = \\
            &= \eta \cdot \eta^{\alpha - 1} \cdot \Ltrans{f}(\eta) - (\Fint{1 - \alpha} f)(0) = \\
            &= \eta^\alpha \cdot \Ltrans{f}(\eta) - (\Fint{1 - \alpha} f)(0).
        \end{aligned}
    \end{equation}
\end{proof}

\begin{lemma}[перетворення Лапласа похідної Катупо]
    \begin{equation}
        \Ltrans{\CFdiff{\alpha} f}(\eta) = \eta^{\alpha} \cdot \Ltrans{f}(\eta) - \sum_{k = 0}^{n - 1} f^{(k)}(0) \cdot \eta^{\alpha - k - 1}.
    \end{equation}
\end{lemma}

\begin{example}
    Зокрема, при $0 < \alpha < 1$ маємо
    \begin{equation}
        \Ltrans{\CFdiff{\alpha} f}(\eta) = \eta^{\alpha} \Ltrans{f}(\eta) - \eta^{\alpha - 1} f(0).
    \end{equation}
\end{example}

\begin{exercise}
    Довести.
\end{exercise}

% \begin{proof}
%     \begin{equation}
%         \begin{aligned}
%             \Ltrans{\CFdiff{\alpha} f}(\eta) &= \Ltrans{\Fint{1 - \alpha} \frac{\diff f}{\diff t}}(\eta) = \\
%             &= \eta^{\alpha - 1} \cdot \Ltrans{\frac{\diff f}{\diff t}}(\eta) = \\
%             &= \eta^{\alpha - 1} \cdot (\eta \cdot \Ltrans{f}(\eta) - f(0) ) = \\
%             &= \eta^{\alpha} \cdot \Ltrans{f}(\eta) - \eta^{\alpha - 1} \cdot f(0).
%         \end{aligned}
%     \end{equation}
% \end{proof}

\subsubsection{Теорема Таубера і наслідок з неї}

\begin{lemma}[перетворення Лапласа сталої]
    $\Ltrans{c}(\eta) = c / \eta$.
\end{lemma}

\begin{lemma}[перетворення Лапласа множника-експоненти]
    $\Ltrans{e^{p t} f(t)}(\eta) = \Ltrans{f}(\eta - p)$.
\end{lemma}

Нагадаємо, що раніше ми з'ясували, що
$\Ltrans{t^{-\beta}}(\eta) = \Gamma(1 - \beta) \cdot \eta^{\beta - 1}$.

\begin{theorem}[Таубера]
    Нехай $-\beta > - 1$, $f$ монотонна при великих $t$ (тобто вона монотонна на деякому проміжку вигляду $[t_0, +\infty)$). Тоді $f(t) \sim t^{-\beta}$ при $t \to +\infty$ $\iff$ $\Ltrans{f}(\eta) \sim \Gamma(1 - \beta) \cdot \eta^{\beta - 1}$ при $\eta \downarrow 0$.
\end{theorem}

\begin{remark}
    Тут $f(x) \sim g(x)$ при $x \to x_0$ означає, що $\lim_{x \to x_0} \frac{f(x)}{g(x)} = 1$.
\end{remark}

\begin{corollary}
    Нехай $0 < \alpha < 1$ і $f$ монотонна при великих $t$, $f \ge 0$ на $[0, +\infty)$ і $\int_0^{+\infty} f(t) \diff t = 1$. Тоді $\forall A > 0$: $f(t) \sim \alpha A \cdot t^{-\alpha - 1}$ при $t \to +\infty$ $\iff$ $\Ltrans{f}(\eta) = 1 - A \cdot \Gamma(1 - \alpha) \cdot \eta^\alpha + o(n^\alpha)$ при $\eta \downarrow 0$.
\end{corollary}

\begin{proof}
    Розглянемо функцію 
    \begin{equation}
        F(t) = \int_t^{+\infty} f(s) \diff s.
    \end{equation}
    Зауважимо, що $F'(t) = -f(t)$.

    \begin{exercise}
        Доведіть, що за наших припущень
        \begin{equation}
            f(t) \sim A \cdot \alpha \cdot t^{-\alpha - 1} \iff F(t) \sim A \cdot t^{-\alpha}.
        \end{equation}
    \end{exercise}
    % \begin{proof}
    %     ($\Longrightarrow$) Якщо $f(t) \sim A \alpha t^{-\alpha - 1}$, то, за визначенням асимптотики, 
    %         \begin{equation}
    %             (\forall \epsilon > 0) \quad (\exists t_0(\epsilon)) \quad (\forall t > t_0) \quad | \frac{f(t)}{A \alpha t^{-\alpha - 1}} - 1 | < \epsilon,
    %         \end{equation}
    %         або ж, що те саме,
    %         \begin{multline}
    %             (\forall \epsilon > 0) \quad (\exists t_0(\epsilon)) \quad (\forall t > t_0) \\
    %             (A - \epsilon) \alpha t^{-\alpha - 1} < f(t) < (A + \epsilon) \alpha t^{-\alpha - 1},
    %         \end{multline}
    %         щоправде вже з іншим $t_0(\epsilon)$, але не суть. \medskip
            
    %         Інтегруємо:
    %         \begin{multline}
    %             (\forall \epsilon > 0) \quad (\exists t_0(\epsilon)) \quad (\forall t > t_0) \\
    %             \int_t^\infty (A - \epsilon) \alpha s^{-\alpha - 1} \diff s < \int_t^\infty f(s) \diff s < \int_t^\infty (A + \epsilon) \alpha s^{-\alpha - 1} \diff s,
    %         \end{multline}
    %         звідки
    %         \begin{multline}
    %             (\forall \epsilon > 0) \quad (\exists t_0(\epsilon)) \quad (\forall t > t_0) \\
    %             (A - \epsilon) t^{-\alpha} < F(t) < (A + \epsilon) t^{-\alpha},
    %         \end{multline}
    %         отримали що хотіли. \medskip
            
    %     % ($\Longleftarrow$) Припустимо, що $F(t) \sim A t^{-\alpha}$, але $f(t) \not\sim A \alpha t^{-\alpha - 1}$. За визначенням, це означає, що  
    %     % \begin{equation}
    %     %     (\exists \epsilon > 0) (\forall t_0) (\exists t > t_0) \quad | \frac{f(t)}{A \alpha t^{-\alpha - 1}} - 1 | > \epsilon.
    %     % \end{equation}
        
    %     % Зрозуміло, що для деякого $C > 0$ нескінченно часто відбувається або
    %     % \begin{equation}
    %     %     \frac{f(t)}{A \alpha t^{-\alpha - 1}} > 1 + C,
    %     % \end{equation}
    %     % або
    %     % \begin{equation}
    %     %     \frac{f(t)}{A \alpha t^{-\alpha - 1}} < 1 - C,
    %     % \end{equation}
    %     % а також нескінченно часто відбувається
    %     % \begin{equation}
    %     %     \frac{f(t)}{A \alpha t^{-\alpha - 1}} = 1.
    %     % \end{equation}
        
    %     % Без обмеження загальності, перше, розглянемо тоді зростаючу $\{t_n\}_{n = 1}^\infty$ таку, що $t_n \to \infty$ при $n \to \infty$ і 
    %     % \begin{equation}
    %     %     \frac{f(t_n)}{A \alpha t_n^{-\alpha - 1}} = 1.
    %     % \end{equation}
        
    %     % Позначимо
    %     % \begin{equation}
    %     %     T_n = \min_{t \ge t_n} \{t: \frac{f(t)}{A \alpha t^{-\alpha - 1}} = 1 + C \}.
    %     % \end{equation}
        
    %     % Для зручності запишемо це як 
    %     % \begin{equation}
    %     %     f(T_n) = (1 + C) A \alpha T_n^{-\alpha - 1}.
    %     % \end{equation}
        
    %     % Без обмеження загальності вважаємо, що $f$ монотонна починаючи з деякого $t_0 < t_1$, а тому $f(T_n) \le f(t_n)$. Тоді
    %     % \begin{equation}
    %     %     (1 + C) A \alpha T_n^{-\alpha - 1} \le A \alpha t_n^{-\alpha - 1}.
    %     % \end{equation}
        
    %     % Логарифмуючи маємо
    %     % \begin{equation}
    %     %     \ln (1 + C) + \ln A + \ln \alpha - (\alpha + 1) \ln T_n \le \ln A + \ln \alpha - (\alpha + 1) \ln t_n,
    %     % \end{equation}
    %     % або ж
    %     % \begin{equation}
    %     %     (\alpha + 1) \ln T_n \ge (\alpha + 1) \ln t_n + \ln (1 + C).
    %     % \end{equation}
        
    %     % Звідси
    %     % \begin{equation}
    %     %     \ln T_n \ge \ln t_n + k,
    %     % \end{equation}
    %     % або ж
    %     % \begin{equation}
    %     %     T_n \ge K t_n,
    %     % \end{equation}
    %     % де $k = \frac{\ln (1 + C)}{\alpha + 1}$ --- додатнє, $K = e^k$. Аналогічним чином можна показати, що $f(t) \ge (1 + C') A \alpha t^{-\alpha - 1}$ на $[T_n/K', T_n]$ для певних сталих $C' < C$ і $1 < K' < K$. \medskip
        
    %     % Розглянемо тепер $\int_{t_n}^{T_n} f(s) \diff s$. Доволі просто показати, що 
    %     % \begin{equation}
    %     %     \begin{aligned}
    %     %         \int_{T_n / K'}^{T_n} f(s) \diff s 
    %     %         &< (A - \epsilon) (T_n/K')^{-\alpha} - (A + \epsilon) T_n^{-\alpha} = \\
    %     %         &= A ( (T_n/K')^{-\alpha} - T_n^{-\alpha} ) - \epsilon ( (T_n/K')^{-\alpha} + T_n^{-\alpha} ).
    %     %     \end{aligned}
    %     % \end{equation}
    %     % для довільного $\epsilon > 0$, починаючи з деякого $n_0(\epsilon)$, звичайно. Але ж $f(t) \ge (1 + C') A \alpha t^{-\alpha - 1}$ на $[T_n/K', T_n]$, а тому
    %     % \begin{equation}
    %     %     \begin{aligned}
    %     %         \int_{T_n/K'}^{T_n} f(s) \diff s 
    %     %         &\ge \int_{T_n/K'}^{T_n} (1 + C') A \alpha t^{-\alpha - 1} = \\
    %     %         &= (1 + C') A (T_n/K')^{-\alpha} - (1 + C') A T_n^{-\alpha} = \\
    %     %         &= A ( (T_n/K')^{-\alpha} - T_n^{-\alpha} ) + C' A ( (T_n/K')^{-\alpha} - T_n^{-\alpha} ).
    %     %     \end{aligned}
    %     % \end{equation}
        
    %     % Лишилося порівняти (якщо права частина більше то ми отримали протиріччя)
    %     % \begin{equation}
    %     %     \epsilon ( (T_n/K')^{-\alpha} + T_n^{-\alpha} ) \lor C' A ( (T_n/K')^{-\alpha} - T_n^{-\alpha} ),
    %     % \end{equation}
    %     % або ж
    %     % \begin{equation}
    %     %     \epsilon T_n^{-\alpha} K_2 \lor C' A T_n^{-\alpha} K_3,
    %     % \end{equation}
    %     % де $K_2 = (K')^\alpha + 1$, $K_3 = (K')^\alpha - 1$ --- додатні константи. \medskip
        
    %     % Помітимо, що тепер $T_n^{-\alpha}$ можна скоротити, отримаємо
    %     % \begin{equation}
    %     %     \epsilon K_2 \lor C' A K_3.
    %     % \end{equation}
        
    %     % Цілком очевидно, що при $\epsilon \to 0$ права частина переважає, а тому отримали протиріччя.
    % \end{proof}

    Розглянемо
    \begin{equation}
        \Ltrans{F}(\eta) = \int_0^{+\infty} e^{-\eta t} \int_t^{+\infty} f(s) \diff s \diff t,
    \end{equation}
    % картинка Тоді, з
    Змінюючи порядок інтегрування, отримуємо
    \begin{equation}
        \begin{aligned}
            \int_0^{+\infty} f(s) \int_0^s e^{-\eta t} \diff t \diff s
            &= \int_0^{+\infty} f(s) \cdot \frac{1 - e^{-\eta s}}{\eta} \diff s = \\
            &= \frac{1}{\eta} \left( 1 - \int_0^{+\infty} f(s) e^{-\eta s} \diff s \right) = \\
            &= \frac{1 - \Ltrans{f}(\eta)}{\eta}.
        \end{aligned}
    \end{equation}

    Звідси
    \begin{equation}
        \Ltrans{f}(\eta) = 1 - \eta \cdot \Ltrans{F}(\eta).
    \end{equation}

    Тепер можемо записати
    \begin{equation}
        \begin{aligned}
            f(t) \underset{t \to \infty}{\sim} A \cdot \alpha \cdot t^{-\alpha - 1}
            &\iff F(t) \underset{t \to \infty}{\sim} A \cdot t^{-\alpha} \iff \\
            &\iff \Ltrans{F}(\eta) \underset{\eta \to 0+}{\sim} A \cdot \Gamma(1 - \alpha) \cdot \eta^{\alpha - 1} \iff \\
            &\iff \Ltrans{F}(\eta) = A \cdot \Gamma(1 - \alpha) \cdot \eta^{\alpha - 1} + o(\eta^{\alpha - 1}) \iff \\
            &\iff \Ltrans{f}(\eta) = 1 - A \cdot \Gamma(1 - \alpha) \cdot \eta^{\alpha} + o(\eta^{\alpha}).
        \end{aligned}
    \end{equation}
\end{proof}

\section{Моделі аномальної дифузії}

\subsubsection{Основні поняття}

Нагадаємо, що класичні рівняння дифузії та теплопровідності зумовлені наступними чинниками:
\begin{enumerate}
    \item закон збереження кількості речовини/тепла;
    \item джерела і стоки;
    \item ``закон'' Фіка/Фур'є --- емпірично встановлене для широкого класу процесів твердження про те, що інтенсивність потоку речовини/тепла пропорційна мінус градієнту концентрації речовини/кількості тепла на границі.
\end{enumerate}

Розглянемо тепер інший підхід, який грунтується на випадкових блуканнях з неперервним часом (\textit{eng.} CTRW, continuous time random walk). А саме, нехай $x(t)$ --- випадкова величина (координата частинки в момент часу $t$), а  $u(x, t)$ (при фіксованому $t$ --- щільність координати частинки). \medskip

% картинка

Параметри моделі:
\begin{enumerate}
    \item $u_0(x)$ --- щільність початкового (при $t = 0$) розподілу;
    \item $\psi(t)$ --- щільність часу очікування наступного стрибка;
    \item $\lambda(x)$ --- щільність зміщення.
\end{enumerate}

\begin{definition}
    Нехай $f: \RR \to \RR$, тоді її \textit{перетворенням Фур'є} називається 
    \begin{equation}
        \Ftrans{f}(\omega) = \tilde{f}(\omega) = \int_\RR e^{i \omega t} f(x) \diff x.
    \end{equation}
\end{definition}

\begin{proposition}
    Для перетворення Фур'є справедлива теорема згортки:
    \begin{equation}
        \Ftrans{f \star g}(\omega) = \Ftrans{f}(\omega) \cdot \Ftrans{f}(\omega),
    \end{equation}
    де
    \begin{equation}
        (f \star g)(x) = \int_\RR f(y) g(x - y) \diff y.
    \end{equation}
\end{proposition}

\begin{definition}
    Нехай $u(x, t): \RR \times \RR \to \RR$, тоді її \textit{перетворенням Фур'є-Лапласа} називається 
    \begin{equation}
        \FLtrans{u}(\omega, \eta) = \tilde{\bar{u}}(\omega, \eta) = \int_0^{+\infty} \int_\RR e^{-\eta t + i \omega x} u(x, t) \diff x \diff t.
    \end{equation}
\end{definition}

\subsubsection{Формула Монтрола-Вайса}

\begin{th_formula}[Монтрола-Вайса]
    \begin{equation}
        \FLtrans{u}(\omega, \eta) = \frac{\Ftrans{u_0}(\omega)}{\eta} \frac{1 - \Ltrans{\psi}(\eta)}{1 - \Ltrans{\psi}(\eta) \cdot \Ftrans{\lambda}(\omega)},
    \end{equation}
    як тільки
    \begin{equation}
        |\Ltrans{\psi}(\eta) \cdot \Ftrans{\lambda}(\omega)| < 1.
    \end{equation}
\end{th_formula}
\begin{proof}
    Введемо додаткове позначення: $n(t)$ --- кількість стрибків до моменту $t$. $\psi_k(t)$ --- щільність часу $k$-го стрибка. І нарешті, $\lambda_k(x)$ --- щільність координати після $k$-го стрибка. \medskip

    Запишемо формулу повної ймовірності:
    \begin{equation}
        \begin{aligned}
            u(x, t)
            &= \sum_{k = 0}^\infty \mathsf{P}\{n(t) = k\} \lambda_k (x) = \\
            &= \sum_{k = 0}^\infty (\mathsf{P}\{n(t) \ge k\} - \mathsf{P}\{n(t) \ge k + 1\}) \lambda_k (x) = \\
            &= \sum_{k = 0}^\infty \left( \int_0^t \psi_k(s) \diff s - \int_0^t \psi_{k + 1} (s) \diff s \right) \lambda_k (x) = \\
            &= \sum_{k = 0}^\infty \left( \int_0^t \psi^{\star k}(s) \diff s - \int_0^t \psi^{\star(k + 1)} (s) \diff s \right) \cdot (u_0 \star \lambda^{\star k}) (x).
        \end{aligned}
    \end{equation}

    \begin{remark}
        Тут ми скористалися тим, що 
        \begin{equation}
            \psi_k \equiv \psi^{\star k} \equiv \underset{k}{\underbrace{\psi \star \psi \star \ldots \star \psi}}
        \end{equation}
        і
        \begin{equation}
            \lambda_k \equiv u_0 \star \lambda^{\star k} \equiv u_0 \star \underset{k}{\underbrace{\lambda \star \lambda \star \ldots \star \lambda}}.
        \end{equation}
    \end{remark}
    
    \begin{proof}
        Справді, аби здійснити $k$-ий стрибок у момент часу $T$ необхідно зробити $k-1$ стрибок до моменту часу $t$, після чого зачекати час $T - t$. Вже видно, звідки береться згортка у першій формулі, лишається просто формально застосувати математичну індукцію по $k$. \medskip
    
        Справді, потрапити у точку $y$ після $k$ стрибків можна з довільної точки $x$ у якій ми опинилися після $(k - 1)$-го стрибка, причому ймовірність цього $\lambda (y - x)$. Вже видно, звідки береться згортка у другій формулі, лишається просто формально застосувати математичну індукцію по $k$.
    \end{proof}

    Враховуючи, що
    \begin{itemize}
        \item $\Ltrans{\Fint{1} f}(\eta) = \frac{1}{\eta} \Ltrans{f}(\eta)$;
        \item $\Ltrans{\psi^{\star k}}(\eta) = (\Ltrans{\psi}(\eta))^k$;
        \item $\Ftrans{\lambda^{\star k}}(\omega) = (\Ftrans{\lambda}(\omega))^k$;
    \end{itemize}
    маємо
    \begin{equation}
        \begin{aligned}
            \FLtrans{u}(\omega, \eta) 
            &= \sum_{k = 0}^\infty \frac{1}{\eta} ((\Ltrans{\psi}(\eta))^k - (\Ltrans{\psi}(\eta))^{k + 1}) \cdot \Ftrans{u_0}(\omega) \cdot (\Ftrans{\lambda}(\omega))^k = \\
            &= \frac{\Ftrans{u_0}(\omega) \cdot (1 - \Ltrans{\psi}(\eta))}{\eta} \sum_{k = 0}^\infty (\Ltrans{\psi}(\eta))^k \cdot (\Ftrans{\lambda}(\omega))^k = \\
            &= \frac{\Ftrans{u_0}(\omega) \cdot (1 - \Ltrans{\psi}(\eta))}{\eta (1 - \Ltrans{\psi}(\eta) \cdot \Ftrans{\lambda}(\omega)},
        \end{aligned}
    \end{equation}
    де останній перехід --- формула для суми нескінченної геометричної прогресії.
    
    \begin{remark}
        Рівності
        \begin{equation}
            \Ltrans{\psi^{\star k}}(\eta) = (\Ltrans{\psi}(\eta))^k
        \end{equation}
        і
        \begin{equation}
            \Ftrans{\lambda^{\star k}}(\omega) = (\Ftrans{\lambda}(\omega))^k
        \end{equation}
        --- безпосередній наслідок теореми згортки для перетворень Лапласа та Фур'є відповідно, а також застосування методу математичної індукції по $k$.
    \end{remark}
\end{proof}

\begin{equation}
    \ol{\delta^2(t, T)}
    = \frac{1}{T - t} \int_0^{T - t} \Big( x(t + s) - x(s) \Big)^2 \diff s
\end{equation}
--- усереднене зміщення (за часом). Усереднимо за сукупністю частинок:
\begin{equation}
    \begin{aligned}
        \ol{\Blangle \qq^2(t, T) \Brangle}
        &= \frac{1}{T - t} \int_0^{T - t} \Blangle x(t + s) - x(s) \Brangle \diff s = \\
        &= \frac{1}{T - t} \int_0^{T - t} 2 \sigma^2 \Blangle n(t + s) - n(s) \Brangle \diff s.
    \end{aligned}
\end{equation}

За умов, що $t \to \infty$, $T \to \infty$ і $T \gg t$ маємо
\begin{equation}
    \langle n(t) \rangle
    \sim \frac{1}{\tau^\alpha \Gamma(1 + \alpha)} t^\alpha,
\end{equation}
а тому попередній вираз асимптотично рівний
\begin{equation}
    \frac{2 \sigma^2}{T - t} \int_0^{T - t} \frac{1}{\tau^\alpha \Gamma(1 + \alpha)} \Big( (s + t)^\alpha - s^\alpha \Big) \diff s.
\end{equation}

У свою чергу, можемо переписати $(s + t)^\alpha - s^\alpha$ за рядом Тейлора:
\begin{equation}
    (s + t)^\alpha - s^\alpha
    = s^\alpha \left( 1 + \frac{t}{s} \right)^\alpha - s^\alpha
    \sim s^\alpha \left( 1 + \frac{\alpha t}{s} \right) - s^\alpha
    = \frac{\alpha t}{s^{1 - \alpha}},
\end{equation}
а тому попередній вираз асимптотично рівний
\begin{equation}
    \begin{aligned}
        \frac{2 \sigma^2}{T - t} \cdot \frac{\alpha t}{\Gamma(1 + \alpha) \tau^\alpha} \int_0^{T - t} s^{\alpha - 1} \diff s
        &= \frac{2 \sigma^2 \alpha t}{(T - t) \Gamma(1 + \alpha) \tau^\alpha} \cdot \frac{1}{\alpha} \cdot (T - t)^\alpha
        &= \frac{2 \sigma^2 t}{\Gamma(1 + \alpha) \tau^\alpha} (T - t)^{\alpha - 1}
        &\sim \frac{2 K_\alpha}{\Gamma(1 + \alpha) T^{1 - \alpha}} t,
    \end{aligned}
\end{equation}
тобто отримали лінійну функцію від $t$. \medskip

\begin{definition}
    Ситуація, у якій $\Blangle x^2(t) \Brangle$ і $\ol{\Blangle \delta^2(t, T) \Brangle}$ мають рязний вигляд фу функції змінної $t$ називається \textit{слабкою неергодичністю} (\textit{eng.} weak ergodicity breaking).
\end{definition}

\begin{remark}
    В обмеженій області
    \begin{align}
        \Blangle x^2(t) \Brangle &= \const, \quad t \to \infty, \\
        \Blangle \delta^2(t, T) \Brangle &= \const \cdot t^{1 - \alpha}, \quad t \to \infty.
    \end{align}
\end{remark}

Розглянемо тепер ситуацію коли $0 < \alpha < 1$ і $\psi(t) \sim A \alpha t^{-1 - \alpha}$, тоді математичне сподівання $\int_0^\infty t \psi(t) \diff t = \infty$ --- так званий \textit{розподіл з важким хвостом} (\textit{eng.} fat-tailed distribution).

\begin{example}
    Нехай $\psi(t) = \frac{t}{\tau} e^{-t / \tau}$, тоді $\langle T \rangle = \tau$.
\end{example}

\subsection{Рівняння розподіленого порядку}

Розглянемо випадок коли $\alpha$ --- випадкова величина, розподілена на $(0, 1)$ зі щільністю $p(\alpha)$. Припустимо, що час очікування стрибки задається умовною щільністю
\begin{equation}
    \psi(t|\alpha) \sim A \alpha \frac{\tau^\alpha}{t^{1 + \alpha}},
\end{equation}
де $A = \frac{\alpha}{\sigma (1 - \sigma)}$. \medskip

Тоді
\begin{equation}
    \psi(t) = \int_0^1 \psi(t|\alpha) p(\alpha) \diff \alpha.
\end{equation}

Звідси
\begin{equation}
    \LaplaceTransform{\psi}(\eta)
    = \int_0^\infty e^{-\eta t} \int_0^1 \psi(t|\alpha) p(\alpha) \diff \alpha \diff t.
\end{equation}

Змінюючи порядок інтегрування, отримуємо
\begin{equation}
    \int_0^1 p(\alpha) \int_0^\infty e^{-\eta t} \psi(t|\alpha) \diff t \diff \alpha.
\end{equation}

Далі, $\psi(t) \sim \frac{\alpha}{\Gamma(1 - \alpha)} \frac{\tau^\alpha}{t^{1 + \alpha}}$, а тому, за теоремою Таубера, $\LaplaceTransform{\psi}(\eta) \sim 1 - (\eta \tau)^\alpha + o(\eta^\alpha)$. Замінюючи таким чином внутрішній інтеграл отримуємо
\begin{equation}
    \int_0^1 \Big( 1 - (\eta \tau)^\alpha + o(\eta^\alpha) \Big) \diff \alpha
    \sim 1 - \int_0^1 p(\alpha) (\eta \tau)^\alpha \diff \alpha.
\end{equation}

Позначимо
\begin{equation}
    I(\eta, \tau) = \int_0^1 p(\alpha) (\eta \tau)^\alpha \diff \alpha.
\end{equation}

За формулою Монтрола-Вайса
\begin{equation}
    \FourierLaplaceTransform{u}(\omega, \eta)
    = \FourierTransform{u_0}(\omega) \cdot \frac{1}{\eta} \cdot \frac{1 - \LaplaceTransform{\psi}(\eta)}{1 - \LaplaceTransform{\psi}(\eta) \FourierTransform{\lambda}(\omega)}
    = \frac{\FourierTransform{u_0}(\omega)}{\eta} \cdot \frac{I(\eta, \tau)}{1 - (1 - I(\eta, \tau)) \FourierTransform{\lambda}(\omega)}.
\end{equation}

Також припустимо, що $\lambda \sim \mathcal{N}(0, 2 \sigma^2) \implies \FourierTransform{\lambda} \sim 1 - \sigma^2 \omega^2$. Тоді
\begin{equation}
    \FourierLaplaceTransform{u}(\omega, \eta)
    = \frac{\FourierTransform{u_0}(\omega)}{\eta} \cdot \frac{I(\eta, \tau)}{I(\eta, \tau) + \sigma^2 \omega^2}.
\end{equation}

Отже, з точністю до малих доданків,
\begin{equation}
    \FourierLaplaceTransform{u} \cdot \Big( I(\eta, \tau) + \sigma^2 \omega^2 \Big) = \frac{\FourierTransform{u_0}}{\eta} \cdot I(\eta, \tau),
\end{equation}
або ж
\begin{equation}
    I(\eta, \tau) \left( \FourierLaplaceTransform{u} - \FourierTransform{u_0}{\eta} \right) = -\sigma^2 \omega^2 \FourierLaplaceTransform{u}.
\end{equation}

Діємо на це співвідношення оберненим перетворенням Фур'є, отримаємо
\begin{equation}
    I(\eta, \tau) \left( \LaplaceTransform{u} - \frac{u_0}{\eta} \right) = -\sigma^2 \cdot \frac{\partial^2 \LaplaceTransform{u}}{\partial x^2}.
\end{equation}

\begin{remark}
    Тут $\LaplaceTransform{u} = \LaplaceTransform{u}(x, \eta)$.
\end{remark}

Розпишемо інтеграл у явному вигляді:
\begin{equation}
    \begin{aligned}
        I(\eta, \tau) \left( \LaplaceTransform{u} - \frac{u_0}{\eta} \right)
        &= \int_0^1 p(\alpha) (\eta \tau)^\alpha \left( \LaplaceTransform{u} - \frac{u_0}{\eta} \right) \diff \alpha = \\
        &= \int_0^1 p(\alpha) \tau^\alpha \cdot \left( \eta^\alpha \LaplaceTransform{u} - \eta^{\alpha - 1} u_0 \right) \diff \alpha.
    \end{aligned}
\end{equation}

У виразі в дужках під інтегралом не складно впізнати $\LaplaceTransform{{}^\star D_0^\alpha u}(\eta)$. Підставляючи, отримуємо
\begin{equation}
    \int_0^1 p(\alpha) \tau^\alpha \int_0^\infty e^{-\eta t} ({}^\star D_0^\alpha u(x, t)) \diff t \diff \alpha.
\end{equation}

Знову змінюємо порядок інтегрування:
\begin{equation}
    \int_0^\infty e^{-\eta t} \int_0^1 p(\alpha) \tau^\alpha {}^\star D_0^\alpha u(x, t) \diff \alpha \diff t.
\end{equation}

А у цьому, у свою чергу, можна впізнати
\begin{equation}
    \LaplaceTransform{\int_0^1 p(\alpha) \tau^\alpha {}^\star D_0^\alpha u(x, t) \diff \alpha }(\eta).
\end{equation}

Тому, діючи оберненим перетворенням Лапласа на останнє рівняння, отримаємо
\begin{equation}
    \int_0^1 p(\alpha) \tau^\alpha {}^\star D_0^\alpha u(x, t) \diff \alpha = \sigma^2 \frac{\partial^2 u}{\partial x^2},
\end{equation}
\textit{рівняння розподіленого порядку}.

\begin{example}
    Якщо $\alpha = \alpha_0 = \const$, то $p(\alpha) = \delta(\alpha - \alpha_0)$ --- так звана \textit{густина матеріальної точки}:
    \begin{itemize}
        \item $\delta (\alpha - \alpha_0) = 0$, $\forall \alpha \ne \alpha_0$;
        \item $\delta (\alpha_0 - \alpha_0) = \infty$;
        \item $\int_0^1 \delta(\alpha - \alpha_0) \diff x = 1$ ($\alpha_0 \in (0, 1$).
    \end{itemize}    
    
    Тоді отримаємо рівняння
    \begin{equation}
        \tau^{\alpha_0} {}^\star D_0^{\alpha_0} u(x, t) = \sigma^2 \cdot \frac{\partial^2 u}{\partial x^2},
    \end{equation}
    тобто
    \begin{equation}
        {}^\star D_0^{\alpha_0} u(x, t) = K_{\alpha_0} \cdot \frac{\partial^2 u}{\partial x^2}.
    \end{equation}
\end{example}

\begin{example}
    Якщо ж $p(\alpha) \sim \nu \alpha^{\nu - 1}$ при $\alpha \to $ і $x(0) = 0$ (блукання починається з початку координат), то $\langle x^2(t) \rangle \sim c\onst \cdot \ln^\nu t$, так звана \textit{ультраповільна дифузія}.
\end{example}
